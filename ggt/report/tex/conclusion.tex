\section{Conclusion}

Les stratégies non naives telle que TBF ou Complete permette d'explorer efficacement
un graphe de grande taille. Cette efficacité a cependant un prix en terme de complexité
qu'elle soit calculatoire pour Complete ou bien spatiale pour TBF l'efficacité à un cout.
Une question intéressante serait d'améliorer ces stratégies en utilisant les cas d'échecs
d'un test de lien pour transformer ce cas d'echec en une information pertinente pouvant aider
à l'exploration du graphe. Les problématiques de parallélisation sont également riches de
perspectives d'améliorations des performances d'exploration.
