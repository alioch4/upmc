\section{Introduction}

\subsection{Utilisation du programme}

L'utilisation du programme de recherche de communautés permet de trouver
les résultats suivants :

\begin{center}
\begin{tabular}{|c|c|c|}
\hline
  fichier & Communauté & Modularité \\ \hline
  flikr & 49 & 0.3989\\
  flikr-test & 7 & 0.269 \\
\hline
\end{tabular}
\end{center}

Nous pouvons en conclure que les deux graphes de sont pas significativement reliés.
En effet, leur découpage en communauté sont très différents, flikr-test n'est donc
pas une bonne approximation de flikr.

Le second test que nous faisons est un découpage hierarchique, nous obtenons les résultats
suivants :

\begin{center}
\begin{tabular}{|c|c|c|}
\hline
 & flikr-test & flikr \\ \hline
0 & 476 & 27756 \\
1 & 23 & 316 \\
2 & 8 & 49 \\
3 & 7 & - \\
\hline
\end{tabular}
\end{center}


\subsection{Communautés et stratégies de mesure de liens}

Les stratégies TBF et Complete appliquées au graphe de flikr
n'aboutissait pas dans un temps raisonnable
sur la station de travail utilisée pour cette étude. Seul flikr-test
 a été utilisé dans le reste de notre étude pour tester nos programmes.

Des résultats de simulations ayant réussis à terminer sur d'autres stations de travail ont
été consultés afin de voir les observations et analyses effectuées sur flikr-test restent
valides sur flikr.

\subsubsection{Modularité \& Découpage hierarchique}

\begin{center}
\begin{tabular}{|c|c|c|c|c|c|c|}
\hline
Str & Com & Modul& Niv. 0 & 1 & 2 & 3 \\ \hline
random & 31 & 0.694677 & 294 & 112 & 55 & 31 \\
TBF & 7 & 0.401175 & 476 & 23 & 8 & 7 \\
Complete & 7 & 0.401175 & 476 & 23 & 8 & 7 \\
\hline
\end{tabular}
\end{center}


Les valeurs de l pour la stratégie aléatoire a de l'importance à l'évidence.
Plus l sera grand, meilleurs seront les résultats.

Pour les stratégies comme TBF et Complete on peut voir que malgré un nombre
d'essais aléatoire relativement modeste, on arrive à aboutir à des resultats satisfaisant.

En consultant les résultats d'une analyse de flikr faite sur une autre station de travail,
 on constate que le nombre l
 est un facteur plus important pour TBF. La strategie complete s'autoalimente de ses
découvertes, ainsi même si la phase aléatoire n'a pas des résultats très riches,
la phase de recherche intelligente améliorera les résultats pour converger rapidement
vers une solution satisfaisante. Pour TBF il faut que la liste
 de somme de degrés soit préalablement riche. En fonction des résultats obtenus lors de la phase aléatoire
on peut obtenir des explorations plus ou bien fructueuses lors de la phase de recherche intelligente.

Ainsi il faut choisir un l relativement important que TBF puisse fonctionner correctement. Ce qui est
moins important pour Complete.
