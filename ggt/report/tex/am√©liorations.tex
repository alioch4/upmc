\section{Améliorations}

Afin d'améliorer les performances il serait pertinent
de paralléliser les algorithmes d'exploration de façon au augmenter les performances
de nos programmes. En effet, l'exploration peut se paralléliser par exemple, on pourrait
envisager de couper la liste des couples de noeuds à explorer dans la stratégie TBF en
plusieurs jeu de données qui serait repartis sur différentes stations.

De même l'algorithme Complete se parallelise en transformant la liste des noeuds à explorer
en une file d'attente et en transformant les différentes stations en unités de traitement qui prendrait
un élément de la file en entrée puis lancerait une exploration sur l'objet récupéré depuis la file.

Quant à la stratégie random, elle se parallelise trivialement.
