\section{Observations}

La stratégie Random est la plus naive de toutes les stratégies,
elle met un temps très long pour converger et ne donne pas des estimations
très précises de communautés réelles au sein du graphe. Elle a cependant
pour avantage d'avoir une fin prédéterminée, le nombre d'essais que l'on peut tenter.
En outre, son occupation de la mémoire est très restreinte dans la mesure ou elle ne demande
aucune structure suplémentaires (à l'exception de la liste des liens visités).

La stratégie Complete est la plus gourmande des deux stratégies non-naive utilisée.
Sauf cas très improbable on est sur que cette stratégie va finir tôt ou tard par trouver
tous les liens dans le graphe. Elle nécessite une structure de données annexe pouvant prendre
une place importante pour les grands graphes en particulier, si on tombe souvent sur de nouveaux noeuds
la stratégie peut mettre plus de temps à converger.
On remarque également que le nombre de d'essais aléatoires effectués avant de lancer la phase
d'exploration systématique n'a relativement pas d'importance. Pour peu que le nombre soit
de l'ordre de grandeur du nombre de noeuds on constate que la convergence se produit.

La stratégie TBF est un peu plus rapide que la stratégie Complete mais cette rapidité se paye par
une structure annexe qui peut être beaucoup plus lourde (ensemble des couple (u,v)) qui rajoute une
complexité spatiale quadratique en nombre de noeuds.
On remarque que le nombre d'essais aléatoires a une importance sur la vitesse de convergence de
cette stratégie. En effet, il est possible de se dire qu'intuitivement
plus le nombre de liens trouvés est important plus il est probable de trouver des sommes de liens
de valeur importante qui auront plus de chances d'aboutir à des découvertes de liens.
En outre, la stratégie TBF ne peut pas augmenter son reservoir de liens
à tester (à l'inverse d'une stratégie complete).
 A la fin de la phase random on peut connaitre le nombre exact de liens qui seront testés
par la suite. Ainsi, si il y a peu de noeuds à fort degrés dans le reservoir on aura peu de chance de
découvrir tout le graphe.
