\section{Scénarios}

\subsection{Militaire}

Les réseaux de capteurs disposent de propriétés qui intéressent les militaires
: Déploiement rapide, Coût réduit, Auto organisation, Tolérances aux pannes.
Ces critères sont particulièrement pertinents pour des applications de
détection de troupes, de véhicules sur un grand territoire. Il serait également
possible de mettre en place un réseau de capteurs sous-marin
\cite{voigt07sensor} afin d'avoir des informations sur les mouvements suspects
que l'on y trouve.  On pourrait imaginer des cas de tests sur la toxicité ou
les radiations d'une zone avant d'y envoyer une troupe de soldats.  En raison
du caractère confidentiel de ces recherches il est difficile d'avoir une vision
précise de l'état de l'art des applications en production.

En outre, le fait que les réseaux de capteurs soient décentralisés et sans
points critiques rends difficile la neutralisation de ce système.

Les ouvertures dans ce domaines sont très larges. Tout système logistique peut
être adapté dans un certain cadre vers des systèmes militaires. Où se trouve un
certain stock de munition ? Quel est la quantité de signaux ennemis qui sont
détectés ? Aux contraintes classiques des réseaux de capteurs s'ajoute
également des contraintes fortes en terme de sécurité des communications et des
systèmes.

\subsection{Environnement}

La surveillance environnementale est possible via des réseaux de capteurs
\cite{ecology}.  En effet il est possible de déployer sur de grands espaces un
volume important de nœuds pouvant opérer les uns avec les autres et mesurer des
paramètres physiques intéressants.

De nombreuses applications sont déjà en production afin d'atteindre des
objectifs aussi divers que le mouvement des animaux au sein d'un parc animalier
\cite{Cerpa:2001:HMA:371626.371720}, l'étude de la pollution des sols
\cite{Ham92}, ou bien la détection d'incendie de forêts \cite{forest, forest2}.
De nombreuses catastrophes pourraient être évitées via le déploiement de ces
solutions.

\subsection{Sécurité civile}

Des capteurs insérés dans la structure d'un bâtiment tel qu'un barrage
vieillissant pourraient permettre de détecter les problèmes avant qu'une
catastrophe ne se produise. Une autre application pourrait consister à
surveiller les abords d'une voie ferrée ou bien d'une route afin de signaler
aux conducteurs que des animaux peuvent se trouver aux abords des voies.

Plus généralement il est possible de surveiller via divers critères tel que le
son, la vidéo, ou bien la détection de mouvements \cite{Chong03sensornetworks}
et cela dans de nombreux scénarios.

\subsection{Santé}

Le suivi et la veille de l'état de santé d'un patient pourraient se retrouver
considérablement facilités par l'usage de capteurs minuscules.  Qu'ils soient
avalés ou bien insérés sous la peau, ils pourraient permettre de surveiller des
paramètres de santé (Pression artériel, niveau de sucre, \ldots) d'un patient
et d'alerter des médecins en cas de problèmes ou de situation préoccupante
\cite{medical-monitoring, medical-monitoring2, medical-monitoring3,
medical-monitoring4}.

Les problématiques logistiques sont également présentes dans les hôpitaux et
les cliniques.  Les stocks de médicaments, l'approvisionnement des repas en
fonction des traitements particuliers de chacun est une problématique qui
pourrait être considérablement facilitée grace aux réseaux de capteurs
\cite{drug-admin}.

\subsection{Logistique}

Le suivi des colis et des envois de matériel reste un problème difficile,
notamment dans le cas de transits internationaux.  Les réseaux de capteurs
peuvent détecter des puces ou des tags rfid installées dans les étiquettes des
colis et peuvent envoyer des informations à des postes centraux qui se
chargeront de transmettre la localisation du paquet a l'expéditeur et
destinataire via Internet \cite{Kahn99nextcentury, A_wirelessintegrated, pico}.

Les aspects de logistiques sont immenses dans le cas des réseaux de capteurs.
Il serait possible d'imaginer que tous les véhicules et colis sont équipés de
puces pouvant détecter leur présence en continu.  Dans le cas d'inventaire, il
serait pratiquement trivial d'avoir une vision claire de l'état des réserves
d'un magasin, d'un hôpital, d'une bibliothèque,\ldots

En outre, il convient d'observer que cette immense quantité d'informations
pourrait mener à des études statistiques de phénomènes pour le moments
inconnus.  Ainsi en plus de faciliter les problématiques actuelles de
logistique, de nouvelles perspectives pourraient être découvertes via la vision
que ces capteurs pourrait apporter.

\subsection{Ouvertures générales}

Les scénarios d'application des réseaux de capteurs sont complexes, cependant
cette variété de ne doit pas occulter l'aspect le plus fondamental des réseaux
de capteurs: Ils sont une interface entre un phénomène physique et un réseau
informatique. Cet aspect de ``passerelle'' entre des phénomènes physiques et
Internet est leur résumé fonctionnel le plus basique.  Une vision plus fine sur
les réseaux de capteurs consiste à ajouter à cette vision les contraintes
énergétiques et de tirer avantage de leur facilité de déploiement
\cite{dunkels04ercim}.
