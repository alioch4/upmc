\section{Système d'exploitation}

Tous les capteurs ont besoin d'un système d'exploitation pour fonctionner
\cite{dunkels09operating, dutta12operating}.  En plus des solutions
propriétaires il existe plusieurs solutions libres.  Seuls les systèmes
d'exploitation les plus utilisés et maintenus seront présentés par la suite.

\subsection{Tiny OS}

TinyOS est le plus vieux système d'exploitation libre pour capteur encore
maintenu. Il fut inventé à Berkeley en 2000 \cite{Levis04tinyos:an,
Hill00systemarchitecture}. TinyOS repose sur un mécanisme de gestion
d'évènements. Les évènements tel qu'un battement d'horloge, un paquet radio
détecté, la lecture des paramètres d'un capteur sont traités les uns après les
autres au sein d'une file FIFO. Une fois lancées, les taches s'exécutent sans
interruptions.  La latence entre les processus au sein de TinyOS est faible en
effet au moment de lancer un nouveau processus il n'est pas nécessaire
d'allouer de la mémoire. Le système, les applications et les modules utilisés
occupent un volume très compact.  En effet les applications et le système sont
compilés ensemble ce qui donne un système monolithique et une utilisation du CPU
efficace.

 Bien qu'utilisé dans de très nombreux systèmes en raison de son ancienneté, il
n'est pas dispensé de critiques. Pendant longtemps, TinyOS n'avait pas de
supports natif de l'IPv6 et de 6LowPAN. Bien que cet handicap soit aujourd'hui
comblé, de nombreux utilisateurs qui avaient besoin d'IPv6 et de 6LowPAN ont du
trouver une alternative, telle que Contiki afin de disposer de cette
technologie.

La mise à jour de capteurs en production est possible cependant, il est
nécessaire de remplacer toute l'image du système d'exploitation pour y arriver.
Il n'est pas possible de réinstaller simplement une autre application ou d'en
modifier une déjà en production. Plus précisément, l'édition de liens est telle
qu'une mise à jour d'un composant ne peut se faire sans refaire toute l'édition
de liens.

Le noyau de TinyOS n'utilise pas de threads en outre, l'exécution d'une tache
lourde tel que la cryptographie peut comporter des risques important de
dépassement de tampons et mettre en cause l'intégrité de la mémoire du système.

\subsection{LiteOS}

LiteOS \cite{Cao_theliteos} a été crée en 2008 à l'université de l'Illinois. Il
est écrit en C, sous licence GPL et implémente un système d'exploitation de
type UNIX sur un capteur. LiteOS dispose de nombreuses fonctionnalité
intéressantes venue de l'univers UNIX. On peut citer par exemple : Un système
de fichiers hiérarchique, un système de fichier distant pour nœuds du réseau,
ainsi qu'un ensemble de mécanismes hérités du shell. Le noyau de ce système
d'exploitation supporte les threads et peut également faire de la programmation
évènementielle à l'aide d'appel de fonctions.  La mémoire est gérée
dynamiquement et les applications exécutées peuvent être reprogrammées à
distance car le noyau le système de fichier et le shell sont des entités
séparées.

Des critiques peuvent être émises vis à vis de la lourdeur de la norme POSIX
qui est vraiment surdimensionnée pour ce genre de systèmes.


\subsection{Contiki}


Contiki \cite{dunkels04contiki} a été développé en 2004 en Suède.  Le noyau de
ce système d'exploitation est indépendant des applications qui l'utilise. Le
système n'est donc pas monolithique à l'inverse de TinyOS.  Ainsi, il est
possible de mettre à jour une application sans devoir refaire toute une édition
de liens, ce qui simplifie considérablement l'installation de nouvelles
applications. Le système est essentiellement divisé en deux parties : Le
\textit{core} qui contient l'ensemble des bibliothèques, pilotes, runtime,
ordonnanceur ainsi que le noyau du système. La seconde partie tournant dans
l'espace utilisateur contient les programmes des différentes applications en
cours d'exécution sur le système. L'ordonnanceur de Contiki fonctionne par
évènements et n'est pas préemptif ainsi, la création de piles et les mécanismes
de gestion de verrous sur la mémoire est évité. En outre, il est possible pour
une application d'utiliser une bibliothèque tierce afin de disposer de
traitement préemptif si elle en fait explicitement la demande.

Bien qu'étant plus lourd que TinyOS, Contiki dispose d'un noyau disposant de
plus de fonctionnalités tel que la gestion de processus en mode FIFO ou bien le
chargement et le déchargement automatiques des routines nécessaires à
l'exécution d'une application.

\subsection{Remarque}

Il est à noter que des travaux ont été mené pour vérifier l'interopérabilité de
ces systèmes à travers leurs protocoles de communication
\cite{Eriksson09detailedsimulation, Eriksson:2009:CIT:1537614.1537650}. Ainsi
le choix d'un système d'exploitation peut entrainer des pertes de performances
avec un réseau d'ores et déjà existant. Il y a cependant espoir qu'un mouvement
d'interopérabilité se produira entre les différents systèmes d'exploitation
afin d'offrir des implémentations de références de protocoles de communications
n'utilisant aucune pré optimisation liée à une plateforme particulière.

Un parallèle peut être observé entre ce phénomène et les standardisations liées
à l'évolution des différentes implémentations de TCP dans le début de
l'histoire des communications d'Internet et des réseaux informatiques
classiques.
