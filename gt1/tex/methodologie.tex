\section{Simulateurs \& Bancs de tests}

L'ensemble des algorithmes, logiciels et protocoles qui sont développés pour
les réseaux de capteurs doivent être testés avant d'être déployer sur des cas
réels d'utilisation. Plusieurs framework de tests ainsi que des simulateurs ont
été proposés \cite{tsiftes10framework, Bergamini:2010:VWS:1868589.1868611,
eriksson08accurate} afin de mesurer l'efficacité des différentes technologies
mises en œuvre.

\subsection{Senslab}

\subsubsection{État de l'art}

Il existe de nombreux concurrents dans le domaine des tests à grandes échelles
de réseaux de capteurs.  Citons par exemple :moteLab
\cite{Werner-Allen:2005:MWS:1147685.1147769}, Kansei \cite{Arora06kansei:a} ou
TWIST \cite{Handziski:2006:TSR:1132983.1132995}. Ces plateformes imposent
cependant de nombreuses contraintes. Celui d'utiliser pour beaucoup d'entre
elles TinyOS, ce qui en soit bloque l'innovation sur tests qui utilisent
d'autres systèmes d'exploitation. Il est à noter que des tests via ces systèmes
sur les ondes radios ne sont pas également possibles de même que l'injection de
bruits. En outre, pour nombre d'entre elles, la couche MAC est réservée à
l'IEEE 802.15.4.  Ceci a pour effet de couper toute recherche et les
éventuelles optimisations qui pourraient exister au niveau de cette couche.

Développée conjointement par Thales et des universités, la plateforme Senslab
\cite{BURINDESROSIERS-2011-587862, BURINDESROSIERS-2011-599102} a pour but de
tester les logiciels et algorithmes conçus pour les réseaux de capteurs sur des
cas concrets \cite{coapdesign}. Les laboratoires de tests sont répartis dans 4
centres en France à Grenoble, Rennes, Lille et Strasbourg. En outre, les deux
derniers disposent de nœuds mobiles pouvant apporter une plus grande variété
de tests. Aucun système d'exploitation n'est imposé aux utilisateurs qui ont
ainsi une très grande liberté sur les tests qu'ils peuvent lancer. Une fois que
le test est lancé sur la plateforme il est possible de surveiller l'ensemble
des résultats obtenus en temps réel ainsi que de pouvoir d'avoir un contrôle en
temps réel sur la simulation.

\subsubsection{Ouvertures \& Wisebed}

Wisebed \cite{chatzigiannakis09sensappeal} est une plateforme de tests pour
réseau de capteurs. D'origine européenne \cite{fischer08wisebed}, cette
plateforme est accessible pour les utilisateurs disposant d'un compte sur
\url{http://wisebed.eu}. Wisebed se propose d'agir comme une surcouche sur
plusieurs réseaux de tests différents. Ainsi il existe sur cette plateforme une
seule interface pour parler à des simulateurs sans-fils, câblés ou hybrides. Le
but premier de Wisebed est de fédérer au niveau européen de nombreux
simulateurs afin de disposer d'une palette large de bancs de tests. Les
communications qui se feraient via une API unique utiliserait la bibliothèque
Wiselib qui a été publiée \cite{DBLP:journals/corr/abs-1101-3067}.

