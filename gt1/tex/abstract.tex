\begin{abstract}

Déployés en grand nombre dans des environnements difficiles, les réseaux de
capteurs ont besoin de protocoles et de systèmes d'exploitation adaptés à leurs
contraintes. Quelles soient énergétiques ou en terme de puissance de calcul et
de mémoire, les contraintes matérielles nécessitent de remettre en question les
implémentations actuelles des protocoles de communications et de stockage afin
d'aborder un nouveau domaine de l'informatique embarqué.

Cet article a pour objectif de faire un état de l'art sur les points
caractéristiques d'une pile logicielle conçue pour les réseaux de capteurs, des
scénarios de déploiements usuels ainsi que de la méthodologie de tests
préliminaires à de tels déploiements.

\end{abstract}
