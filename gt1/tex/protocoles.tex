\section{Protocoles}

De nombreux protocoles qui régissent Internet aujourd'hui ont été conçu sans
soucis des économies d'énergies.  Dans cette partie nous verrons quelques
protocoles (6LoWPAN, IPv6, \cite{ipv6}, UDP, CoAP \cite{coap},\ldots) qui sont
mis en œuvre au sein des réseaux de capteurs afin d'optimiser les performances
des transmissions au sein d'un environnement sous contraintes.


\subsection{Adressage \& routage}

\subsubsection{IPv6 \& 6LoWPAN}

L'IPv6 bien que connaissant une adoption grandissante n'est pas applicable
directement sur les réseaux de capteurs. En effet, la quantité d'informations
qu'une unité à besoin de transmettre est généralement très réduite et le coût
énergétique engendré par la création d'entête des protocoles IPv6 est trop
important pour un réseau de capteur. Ainsi l'IETF a défini l'IPv6 sur les
réseaux de capteurs sous le nom de 6LoWPAN. Ce protocole se caractérise par sa
courte portée, son taux faible, sa consommation énergétique basse pour une
occupation mémoire faible \cite{Srinivasan06someimplications}.

Ainsi il est nécessaire d'adapter le protocole IPv6 en le simplifiant. En
outre, de la même façon que les protocoles TCP/IP ont connu plusieurs
implémentations au cours de l'histoire, il est pertinent de voir les
performances des communications au sein de réseaux de capteurs adoptant
plusieurs implémentations du même protocole sur plusieurs systèmes
d'exploitation pour capteurs différents.  \cite{dunkels11adhoc, dunkels09ercim,
ipv6}

Il a été montré dans le cas de Contiki et de TinyOs qu'il est possible de les
faire fonctionner ensemble mais qu'une perte de performance se produisait.

\subsubsection{RPL}

RPL est un protocole de routage à vecteur de distance qui est crée et couplé
aux réseaux de capteurs. RPL est construit pour être très flexible. Ce
protocole est proactif et construit ses routes en continue. Le RPL définit un
noeud racine qui va définir les paramètres pour tout le réseau

Adressage et routage le trafic remonte souvent vers un nœud central ou vers
des routes très employées routing protocol for low power and lossy networks
DODAG Destination oriented DAG.

Contiki RPL \& TinyRPL sont deux exemples d'implémentations de RPL qui bien
qu'étant fonctionnelle ont des performances moindres quand elles sont utilisées
l'une avec l'autre \cite{ko11contikirpl} démontrant que les performances du
protocole doivent être prises en compte dès les premières heures. Des détails
d'implémentation peuvent avoir des conséquences lourdes dans les développements
ultérieurs.

\subsection{Accès au ressource - CoAP}

Le \textit{Constrained Application Protocol} (CoAP) s'est rapidement imposé
dans le domaine des réseaux de capteurs comme étant le protocole de référence
au niveau applicatif \cite{coap, coapdesign, kovatsch11low-power}. En effet de
part leur nature, les informations échangées au niveau applicatif sont très
simples. Il s'agit le plus souvent de couples requêtes/réponses sur des
informations envoyées ou reçue par des capteurs/aggrégateur/passerelles. Ainsi,
il est facile de repérer de nombreuses similitudes entre ce type d'échanges et
les échanges se passant sur Internet via le protocole HTTP. En effet, HTTP est
fondamentalement non connecté et dans le cas des réseaux de capteurs il est
clair que maintenir une connexion non utilisée va utiliser des ressources et de
l'énergie pour rien car un capteur doit dormir pendant de très longue période
afin d'économiser son énergie.  Le protocole HTTP peut transporter des
informations de type différentes comme l'audio ou vidéo. Dans le cas des
réseaux de capteurs, les types de données échangés sont moins nombreux on
retrouve surtout le XML et le JSON cependant le modèle requêtes/réponses dans
un paradigme client/serveur persiste et reste un point commun central avec
l'HTTP.

Le protocole de transport utilisé par CoAP est différent de celui utilisé par
HTTP et cela constitue un point de divergence majeur. HTTP utilise TCP pour son
transport cependant, TCP n'est pas adapté pour les réseaux de capteurs, les
entêtes sont jugées trop lourdes et beaucoup trop de transfert de paquet sont
provoqués en raison du mécanisme d'acquittement de paquets. C'est pour cela que
l'on lui préfère UDP \cite{coap-web}. UDP étant basé sur des diagrammes, la tolérances
au pertes est plus importantes et les entêtes sont beaucoup plus courtes.

L'une des fonctionnalité essentielle de CoAP est d'éviter la fragmentation de
paquets et de garder les entêtes aussi petites que possible tout cela bien sur
pour économiser autant d'énergie que possible.

L'orientation des protocoles au niveau applicatif vers le web est voulu. En
effet, l'un des objectifs des réseaux de capteurs est de créer un Internet of
Things ou bien un Web of Things. En permettant à des objets de communiquer les
uns avec les autres et de publier des résultats sur le web, CoAP s'inscrit dans
cet objectif de changer durablement l'interface entre les objets de la vie
courante et le web \cite{Glombitza:2009:IWS:1658192.1658197}.

Le protocole CoAP a été implémenté avec succès dans TinyOS \cite{coap-design}
et dans Contiki \cite{coap} ainsi que via des bibliothèques Python et C.

\subsection{Ouvertures}

La couche applicative permet d'ouvrir sur une très grande gamme d'applications.
L'une d'entre elle pourrait consister à stocker au sein de chaque nœud du
réseau une base de données \cite{tsiftes11database}. L'intérêt de ce genre de
technique pourrait permettre d'avoir des traitements très fins sur les données
qui sont transmises vers l'extérieur du réseau de capteur. Le stockage et
l'organisation de données permettent en effet d'économiser des transmissions
qui sont les plus couteuses du point de vue énergétique.
