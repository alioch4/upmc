\section{Protection contre les pertes}

\begin{figure}[htb]
	\centering
	\includegraphics[width=0.44\textwidth]{gfx/q22a-n.pdf}
	\caption{Taux de perte en fonction de la taille de la file}
	\label{fig:q22a-n}
\end{figure}

L'objectif était de trouver la taille de la file d'attente telle que le taux de
perte soit inférieur à $10^{-4}$.  Les expériences menées mettent en évidence le
lien entre le taux de pertes et la taille du tampon, comme le montre la figure
\ref{fig:q22a-n}. Il est à noter qu'une taille de file de 250 possède une
incertitude bien plus faible que celle pour 200 avec un taux de pertes nettement
inférieur et c'est pour cela que nous avons donc choisi une fille d'attente de
taille 250 pour le reste de nos simulations.

Enfin, pour éviter les pertes consécutives de paquets, ce qui aurait provoqué le
phénomène de synchronisation globale si nous utilisions des applications TCP,
nous allons mettre en place le mécanisme RIO-C, qui est une généralisation du
RED.

\begin{figure}[htb]
	\centering
	\includegraphics[width=0.44\textwidth]{gfx/q22b-rio}
	\caption{Taux de perte après mise en place du RIO-C, en fonction
	du paramètre $\mathrm{min}$}
	\label{fig:q22b-rio}
\end{figure}

La figure \ref{fig:q22b-rio} montre les taux de perte de chaque application en
fonction du paramètre $\mathrm{min}$ du RIO-C.  Le paramètre $\mathrm{max}$,
quant à lui, est fixé à la taille de la file (i.e. 250), et $p_{max}$ est fixé à
1.  On constate une relation évidente entre le paramètre $\mathrm{min}$ et le
taux de perte, bien que le taux de perte du flux de données semble plus
chaotique pour des raisons que nous n'avons pas su expliquer.  Lorsque
$\mathrm{min} \ge 190$, tous les flux ont un taux de perte inférieur à $10^{-4}$.


% vi:tw=80:
