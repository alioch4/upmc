\section{Comparaison de plusieurs mécanismes d'ordonnancement}

On se propose de comparer les mécanismes d'ordonnancement suivants: tête de
file, RR, WRR et PRI, afin de comparer les délais des différents flux.

La figure \ref{fig:q21} montre les délais globaux et des flux individuels en
fonction du mécanisme d'ordonnancement utilisé.  Nous pouvons constater qu'avec
une sporadicité $b = 3$ pour le flux vidéo, que le choix du mécanisme
d'ordonnancement n'a qu'une très faible influence sur les délais.

\begin{figure}[htb]
	\centering
	\includegraphics[width=0.45\textwidth]{gfx/q21}
	\caption{Délais des différents flux en fonction de mécanisme
	d'ordonnancement}
	\label{fig:q21}
\end{figure}

Cependant, la figure \ref{fig:q21a} permet de mettre en évidence deux choses:
non seulement les effets du choix du mécanisme d'ordonnancement sont amplifiés
lorsque la sporadicité $b$ augmente, mais on peut voir également que la
sporadicité du flux vidéo est extrêmement dommageable pour les délais des autres
trafics.  Globalement, l'ordonnancement PRI est le pire, car il est susceptible
de provoquer la famine des autres flux; la discipline tête de file et le WRR
sont un peu plus efficaces, mais l'ordonnancement RR réduit le mieux l'impact
sur les délais lorsque $b$ augmente.

\begin{figure}[htb]
	\centering
	\includegraphics[width=0.45\textwidth]{gfx/q21a}
	\caption{Délais globaux moyens en fonction de la sporadicité du flux
	vidéo}
	\label{fig:q21a}
\end{figure}

Le délai du trafic voix, même lorsque $b = 3$, n'est pas satisfaisant.  C'est
pourquoi on décide de réduire le délai de ce flux, en augmentant le poids du
trafic voix pour le WRR.

La figure \ref{fig:q21b} représente les délais du flux voix en fonction de la
sporadicité de $b$, que nous avons fait varier sur le même intervalle.  Les
poids affectés au WRR sont:
\begin{itemize}
	\item \textbf{Avant}: vidéo + données: 310, voix: 50
	\item \textbf{Après}: vidéo + données: 60, voix: 400
\end{itemize}

On voit que le flux voix était à l'origine négativement impacté par
l'augmentation de $b$.  Après avoir adapté les poids de chaque flux, on constate
qu'on parvient à supprimer entièrement l'influence négative de $b$ et à réduire
légèrement le délai global.

\begin{figure}[htb]
	\centering
	\includegraphics[width=0.45\textwidth]{gfx/q21b}
	\caption{Influence du changement des poids du WRR}
	\label{fig:q21b}
\end{figure}

% vi:tw=80:
