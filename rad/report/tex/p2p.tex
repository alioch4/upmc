\section{Botnet pair-à-pair}

\subsection{Intérêt du modèle pair-à-pair}

Les modèles actuels de botnet sont majoritairement hierarchiques. Les \textit{bot
controllers} sont nécessaires afin de faire fonctionner des botnets présentant des
tailles importantes. Cependant, la configuration d'adresses DNS dynamique peut demander
des ressources temporels et financières importantes. En outre, les fournisseurs d'adresses
DNS dynamiques peuvent en partenariat avec les autorités trouver les comportements suspects
afin de neutraliser des botenets.

Une parade contre cette faiblesse de passage à l'échelle des botnet hierarchique est l'utilisation de botnet pair à pair
ne reposant plus sur une hierarchie mais d'avantage sur des connexions locales.


\subsection{Un vers en deux temps pour construire un botnet pair à pair}

Afin de construire un réseau pair à pair malveillant, un ver peut être utilisé.
Ce ver fonctionne essentiellement en deux temps :
\begin{description}
\item[Vérification :] Cette partie a pour tache que la station qui est en cours de traitement
n'est pas un honeypot. Afin de s'assurer contre cela, le ver peut utiliser une méthodologie similaire
a celle présentée précedement dans l'article. De ce fait, si la station n'est pas digne de confiance
pour le botnet alors le ver n'essaye pas de faire rentrer la station dans le botnet et aucune information
ne sera donné à cette station. Cependant, si la vérification est valide, alors le ver peut passer à la seconde
étape.

\item[Enregistrement :] Cette étape a pour but d'envoyer à la station fraichement corrompue de nouvelles cibles
à aller contaminer avec le même ver qui a été utilisé pour la corrompre elle même. Les listes
en question sont nommées des \textit{buddy list}. L'avantage énorme de ces buddy list réside dans le fait qu'en
cas de prise de controle d'un noeud du botnet par un attaquant. Il est difficile d'avoir accès à plus de noeuds
que ceux contenus dans la buddy list. De cette façon, le réseau est robuste aux attaques. En outre, même si des
noeuds sont attaqués et mis hors circuits par des attaquants. Le botmaster peut toujours donner des ordres via
les noeuds encore connectés.
\end{description}


Ce ver fonctionnant en deux temps rends la création de botnet assez facile et assez resistantes aux honeypots.
