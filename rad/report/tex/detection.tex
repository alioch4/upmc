\section{Détection d'un honeypot}

Les concepteurs de botnets ont plusieurs moyens afin de détecter si
une unité d'un botnet est un honeypot ou bien un zombie digne de confiance.
Dans cette partie nous allons exposer les différentes stratégies que peut employer
un botnet afin de faire le tri entre les unités dignes de confiance et les honeypots.

\subsection{Sonde}

Un outil de prédilection pour trouver des honeypots est une sonde (\textit{probe} en anglais).
Une sonde, est un bot non controlé par un attaquant que le botmaster ou les botcontrollers de confiance
peuvent interroger pour savoir si un message précis à été reçu.

\subsection{Détection par défis d'envoi de contenu malveillant}

En raison de la dimension légale et juridique liée au spam et à l'envoi de fichiers malveillants,
il est clair que les administrateurs système ne peuvent pas contribuer de manière trop zélée au
activités malveillantes du botnet. On tombe ici sur un grave problème, les botnets en testant le
zèle de certains noeuds auront toujours la possibilité de détecter plus ou moins finement les intrus.

De ce fait une stratégie possible pour détecter un honeypot serait de lui envoyer des requêtes comme
``Envoie \numprint{5000} mails contenant des pièces jointes malveillantes aux adresses suivantes''. Si la
station refuse, cela peut signifier qu'elle est momentanément inefficace, ou bien qu'il s'agit d'un honeypot.

Cette stratégie est assez difficile à contrer car l'administrateur du botnet peut
parfaitement mettre en œuvre des sondes parmi les stations (des \textit{probes}, dans
le texte) afin de savoir si une station particulière est de bonne foi
ou non.

Il existe différentes façons d'utiliser cette stratégie.
La première consisterait à utiliser cette stratégie avant de faire rentrer une station dans le botnet.
Il est facile pour un concepteur de botnet de ne donner accès au reste du botnet que si une station
a prouvé son efficacité. Cette méthode a pour avantage de bloquer les honeypots avant qu'ils ne rentrent
dans le botnet.
Une seconde stratégie pourrait consister à remettre en question la confiance que le botnet a dans une station 
de manière périodique. On pourrait placer dans la liste des cibles d'une station une sonde
de confiance aléatoirement choisie et contrôlée par le botnet. Cette stratégie a pour avantage de bloquer les tentatives de prise
de contrôle d'un noeud auxquels le botnet aurait accordé sa confiance.
Ces deux possibilités de vérification sont cumulables.

\subsection{Détection par requête Web}

Une autre façon de détecter si une station est compromise ou non consiste à lui demander d'effectuer
des requêtes HTTP sur une liste de site donnée. Cette procédure est plus difficile à gérer pour un administrateur
sécurité, car il est difficile de savoir si une requête Web est légitime ou bien si elle fait partie d'une tentative
de déni de service distribué. Cette difficulté de faire la différence entre les deux cas rend
la tache de détection des botnets plus difficile.


\subsection{Une lutte difficile}

Il est difficile de savoir si une adresse est un capteur ou non; c'est la raison principale pour laquelle
les honeypots peuvent se faire repérer.
En outre, lorsque une station (honeypot ou non) reçoit une liste de choses à faire, les cibles peuvent
être dissimulés à l'intérieur d'un fichier binaire difficile à analyser afin d'avoir la liste des cibles sans
leur envoyer du contenu. En outre, il est également possible pour un botnet de chiffrer ses communications entre
stations aussi bien que les listes de cibles afin de rendre la rétro-ingénierie plus délicate.

\subsection{Contrôle d'un noeud décideur}

La plupart des administrateurs sécurité hésitent à prendre le contrôle quand ils en ont l'occasion d'un
noeud décideur dans un botnet. Cela pour des raisons juridiques essentiellement. Il est clair que du trafic
partant d'une station est au yeux de la loi la responsable. Ainsi, si une station donneuse d'ordres venait à tomber
sous la responsabilité légale d'un administrateur sécurité, il pourrait être poursuivi pour les dégâts causés.
Ainsi les administrateurs préfèrent souvent éviter de prendre des risques et ne veulent pas prendre le contrôle d'un
noeud donneur d'ordre. Le fait de ne transmettre aucun ordre transforme un bot contrôleur en un ``trou noir'', ou \textit{sinkhole}.
Ce qui le rend bien entendu détectable via un mécanisme de détection à base de sonde comme évoqué précédemment.
Bien que l'utilisation du bot contrôleur pourrait donner de précieuses informations sur le fonctionnement du botnet
(autres stations connectées, liste des tâches en cours, cibles futures, etc.), le risque légal est beaucoup trop
grand et les administrateurs se contentent souvent de mettre sur liste noire les adresses IP répertoriées dans le
bot contrôleur et de ne pas prendre d'autres mesures.
