\section{Corpus existant}

Les botnets ont toujours été l'une des plus grande cause de cybercriminalité sur
Internet. Ils véhiculent de nombreux type de programmes malveillants, sont responsables
de nombreux DDoS et peuvent capturer de nombreux mots de passe en utilisant des mécanismes
de keylogging. De ce fait, ils sont étudiés par de nombreux chercheurs en securité informatique.

Deux tendances majeures se distinguent dans la lutte contre les botnets.
Le premier est l'études des ``honeypots''. Les honeypots sont construits par les
personnes en charge de la securité informatique afin de mener des contre offensives
contre les botnets. En faisant passer un programme pour une machine normale au sein
d'un botnet, des informations intéressantes pour la lutte contre ce dernier peuvent
être extraites et des contre mesures peuvent être appliquées.

Une autre solution consiste à utiliser les mécanismes de DNS dynamiques
afin qu'une machine controllée par un attaquant du botnet puisse se faire
passer pour un controlleur du botnet en son sein. De cette façon des ordres
allant à l'encontre des intérêts du botnet peuvent être executées et le botnet
peut perdre le contrôle d'une partie de ses machines \cite{botnet-timezone}.


Les ``honeypots'' sont étudiés activement par les chercheurs
(\cite{honeypot-framework}, \cite{honeypot-hybrid}) et professionnels de
la securité informatique, car ils permettent de surveiller et défendre
un réseau informatique contre de nombreuses attaques informatiques.
