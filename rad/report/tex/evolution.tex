\section{Travaux ultérieurs}

Un modèle pair à pair n'est pas nécessairement le plus adapté pour
gérer un botnet. Cependant un modèle hybride permet de combiner les
avantages du modèle hiérarchique et du modèle distribué. Cette partie
est une ouverture vers les travaux plus récents qui porte l'influence
des idées de l'article de Zou et Cunningham \cite{main}.

\subsection{Présentation du modèle hybride}

Le modèle pair à pair est très flexible. Il existe de nombreuses différences
entre les modèles utilisés par Napster, BitTorrent, ou d'autres protocoles.
L'un des intérêts d'utiliser le modèle pair à pair est d'augmenter la robustesse
du botnet en cas d'attaque. Il est clair que tous les modèles de pair à pair ne
peuvent pas convenir. Un modèle pair à pair centralisé, tel que celui utilisé
par Napster, ne pourrait pas fonctionner car il suffirait comme dans le cas
distribué de neutraliser le serveur principal pour arrêter le botnet.

\subsection{Amélioration du mécanisme de construction du botnet}

L'une des améliorations des travaux ultérieurs est le mécanisme de réinfection
\cite{p2p}.  Au moment de la construction d'un botnet, un noeud A peut choisir
de recontaminer un noeud B d'ores et déjà intégré au botnet. Pour ce faire A
envoie un B une liste d'hôtes qu'il doit intégrer dans sa liste de pairs de
façon aléatoire. De cette manière, dans l'hypothèse où un noeud serait capturé,
il serait très difficile pour un attaquant du botnet de retrouver l'historique
d'infection d'un noeud particulier. Cette amélioration rend le botnet plus
difficile à mesurer pour ses attaquants.

\subsection{Nouvelles parades}

La robustesse des botnets hybrides se base sur le fait que les listes de pairs se
mettent à jour régulièrement et se propagent rapidement. En laissant un honeypot
se faire infecter puis infecter un maximum d'honeypots, il est possible
d'attaquer le botnet en faisant croire au contrôleur du botnet que les
honeypots peuvent être des servants et récupérer des adresses IP d'hôtes
infectés. Cette stratégie est d'autant plus efficace si les attaquants du botnet
disposent de grandes plages d'adresses IP.


\subsection{Ouvertures}

Le modèle pair à pair prendra plus d'importance dans les années à venir en
raison de ses nombreux avantages. La robustesse de ce modèle le rend
particulièrement résistant aux attaques classiques. Une parade face à ce genre
de botnet peut être trouvée via l'utilisation des honeypots afin de trouver des
informations pertinentes vis à vis de ces systèmes. L'étude des honeypots doit
devenir un domaine d'étude de premier plan pour lutter contre les botnets.

