\section{Introduction}

\subsection{Définitions}

Un ``botnet'' est un réseau d'ordinateurs exécutant des programmes malveillants
dans le but d'arriver à des objectifs communs comme les dénis de service distribués,
de la diffusion de spam, ou bien des attaques de type ``brute force'' contre
des mots de passe. Les ordinateurs effectuant ces tâches sont appelés des
``zombies'', et l'entité qui les contrôle porte le nom de ``botmaster''.

Un ``honeypot'' est un moyen de faire de la détection ou de la surveillance de
botnet. C'est généralement un programme qui a la propriété de détecter et
d'attirer les programmes malveillants vers lui afin de les piéger.  Ce
programme tourne sur une machine qu'on laisse délibérément faire partie d'un
botnet, mais qui sont laissés sous le contrôle d'un administrateur système,
d'un responsable sécurité, ou d'un expert similaire, que l'on appellera
\textit{attaquant}.  Les honeypots ont été utilisés avec succès ces dernières
années afin de mener des attaques contre les botnets. Leur étude et leur
perfectionnement est une menace pour la croissance des botnets sur Internet et
explique pourquoi les administrateurs de botnets essayent de rendre leurs
botnets plus résistants aux honeypots.

\subsection{Enjeux}

Les enjeux de la lutte contre les botnets sont aujourd'hui de plus en plus
importants. En effet, ces réseaux sont généralement associés à de vastes
campagnes d'envoi de spams, de distribution de malware, de cybercriminalité,
voire d'attaques par déni de service distribué (\textit{Distributed Denial of
Service}, DDoS).  La taille parfois importante des botnets en font des
``armées'' particulièrement intéressantes à exploiter à des fins
malveil\-lantes.

Idéalement, les honeypots, de leur côté, devraient être indépendantes de la
plate-forme logicielle ou matérielle, et ne doivent pas non plus engager la
responsabilité légale des attaquants, qui font généralement partie d'une
société spécialisée.  D'un côté, une attaque, un spam ou une tentative
d'intrusion ne doit préférablement pas provenir de la société opératrice, sous
peine de poursuites judiciaires. De l'autre côté, le propriétaire du botnet
peut mettre en œuvre des mécanismes de détection de honeypots au sein de son
botnet, d'où un exigence pour un honeypot de devoir se fondre dans la masse.
Dans de tels cas, un honeypot qui se fait repérer est rapidement éjecté du
botnet, et devient donc complètement inefficace.

\subsection{Plan de la fiche de lecture}

Dans cette fiche de lecture, nous présenterons tout d'abord, en introduction,
les enjeux liés aux honeypots et aux botnets. Nous verrons les défenses que les
administrateurs de botnets peuvent mettre en place afin de lutter contre les
honeypots, quelles sont les limites de ces défenses, et dans quelle mesure il
est possible de les contourner. Nous verrons ensuite les apports de l'article à
ce domaine, puis les travaux récents qui ont été publiés dans ce domaine et
comment ils se raccordent entre eux.

