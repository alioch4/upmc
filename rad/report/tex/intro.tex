\chapter*{Introduction}
\addcontentsline{toc}{chapter}{Préface}


Dans cette fiche de lecture, nous présenterons tout d'abord, en introduction, les enjeux liés aux honeypots et aux
botnets. Nous verrons les défenses que les créateurs de botnets peuvent mettre en place afin
de lutter contre les honeypots, quelles sont les limites de ces défenses, et dans quelle mesure il est possible 
de les contourner. Nous verrons ensuite les apports de l'article à ce domaine, puis les travaux récents
qui ont été publiés dans ce domaine et comment ils se raccordent entre eux.

Les enjeux sont importants : les honeypots ici présentés servent à repérer les botnets.  Ceux-ci sont généralement associés à de vastes campagnes de spam, de distribution de malware, de cybercriminalité, voire d'attaques par déni de service distribué (\textit{Distributed Denial of Service}, DDoS).  La taille parfois importante des botnets en font des ``armées'' particulièrement intéressantes à exploiter à de telles fins.

Idéalement, des telles solutions devraient être indépendantes de la plate-forme logicielle ou matérielle, et ne doivent pas non plus engager la responsabilité légale des opérateurs de honeypots.  D'un côté, une attaque, un spam ou une tentative d'intrusion ne doit préférablement pas provenir de la société opératrice, sous peine de poursuites judiciaires. De l'autre côté, le propriétaire du botnet pourrait très bien mettre en œuvre des mécanismes de détection de honeypots au sein de son botnet, d'où un exigence pour un honeypot de devoir se fondre dans la masse.  Dans de tels cas, un honeypot qui se fait repérer est rapidement éjecté du botnet, et devient donc complètement inefficace.


Il faut utiliser des références :
Propagation de virus : 6
Enjeux : 21
