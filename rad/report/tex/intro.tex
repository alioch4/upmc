\chapter*{Introduction}
\addcontentsline{toc}{chapter}{Préface}


Dans cette fiche de lecture, nous présenterons tout d'abord, en introduction, les enjeux liés aux honeypots et aux
botnets. Nous verrons les défenses que les créateurs de botnets peuvent mettre en place afin
de lutter contre les honeypots, quels sont les limites de ces défenses et dans quelle mesure il est possible 
de les contourner. Nous verrons ensuite les apports de l'article à ce domaine puis quels sont les travaux récents
qui ont étés publiés dans ce domaine et comment ils se raccordent entre eux.



Les enjeux :
malwares
spam
DDoS
Il faut mettre en évidence que c'est vraiment quelque chose de très délicat.

Trouver des solutions qui sont indépendantes de la plate-forme soft ou hard sur laquelle on se trouve.

Mettre en évidence les contraintes légales. On ne peux pas spammer comme un bourrin pour faire croire au Botnet qu'on est un
méchant. Il faut trouver des parades.


Il faut utiliser des références :
Propagation de virus : 6
Enjeux : 21

Pour fonctionner un honeypot a besoin de se fondre dans la masse des ordinateurs zombies d'un botnet.
Un honeypot qui se fait reperé est rapidement ejecté du botnet et devient donc complètement inefficace.
