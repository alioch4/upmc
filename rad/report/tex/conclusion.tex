\section*{Conclusion générale}

Nous avons vu dans ce papier quelles étaient les techniques pour construire
des botnets résistants aux honeypots. Le constat est assez clair: si
les attaquants ne prennent pas le contrôle de nœuds du botnet,
ils ne disposent de peu d'informations et de peu de moyens pour agir efficacement
et durablement pour les neutraliser.

Cet article sort également de sa dimension strictement technique pour toucher un
problème de fond, qui est celui de la responsabilité juridique des attaquants.
Tant qu'ils n'auront pas les mains plus libres pour essayer de s'infiltrer dans
le botnet, quitte à envoyer des flux malveillants, les administrateurs de
botnets auront toujours une longueur d'avance, et les attaquants seront toujours
condamnés à se défendre de manière non offensive contre les botnets.  Une
institution non gouvernementale et internationale pourrait permettre d'encadrer
ces cas houleux afin d'avoir à terme une méthodologie efficace contre les
botnets, qui seront toujours libres d'agir à l'encontre de tout impératif légal.
