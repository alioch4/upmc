\chapter*{Conclusion générale}
\addcontentsline{toc}{chapter}{Conclusion}


Nous avons vu dans ce papier quelles étaient les techniques pour construire
des botnets resistants contre les honeypot. Le constat est assez clair, si
les administrateurs systèmes ne prennent pas le contrôle de noeud du botnet,
ils ne disposent de peu d'informations et de peu de moyens pour agir efficacement
et durablement pour la neutralisation des botnets.

Cet article sort également de sa dimension strictement technique pour toucher
un problème de fond de responsabilité judiciaire des administrateurs. Tant
que les administrateurs de parc informatique n'auront pas les mains plus libres
pour essayer de rentrer dans le botnet quitte à envoyer des flux malveillants, les
controlleurs de botnet auront toujours une longueur d'avance et les administrateurs
seront toujours condamnés à avoir une defense non offensive contre les botnet. Une
institution non gouvernementale et internationale pourrait permettre d'encadrer ces
cas houleux afin d'avoir à terme une methodologie efficace contre les botnets qui seront
toujours libres d'agir à l'encontre de tout impératif légaux.
