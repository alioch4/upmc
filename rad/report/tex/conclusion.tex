\section*{Conclusion générale}

Nous avons vu dans ce papier quelles étaient les techniques pour construire
des botnets résistants aux honeypots. Le constat est assez clair: si
les administrateurs système ne prennent pas le contrôle de noeuds du botnet,
ils ne disposent de peu d'informations et de peu de moyens pour agir efficacement
et durablement pour neutraliser les botnets.

Cet article sort également de sa dimension strictement technique pour toucher
un problème de fond, qui est celui de la responsabilité juridique des
administrateurs. Tant que les administrateurs de parc informatique n'auront pas
les mains plus libres pour essayer de rentrer dans le botnet, quitte à envoyer
des flux malveillants, les contrôleurs de botnet auront toujours une longueur
d'avance et les administrateurs seront toujours condamnés à se défendre de manière
non offensive contre les botnets. Une institution non gouvernementale et
internationale pourrait permettre d'encadrer ces cas houleux afin d'avoir à
terme une méthodologie efficace contre les botnets qui seront toujours libres
d'agir à l'encontre de tout impératif légaux.
