\section{Anatomie d'un botnet hierarchique}

Les botnets sont controllés dans la majorité des cas par un
réseau hierarchique. Un serveur IRC est souvent utilisé pour
permettre au botmaster d'envoyer des ordres aux postes contaminés.
Cependant dans une topologie hierarchique, ce serveur est critique.
Si il tombe, le botnet ne peut plus recevoir d'ordres et devient donc
inopérant.

Afin de ne pas avoir à discuter avec des millions de machines à la fois,
le botmaster peut déléguer les envois d'ordres à une groupe de station infectées
appellées \textit{bot controller}. Ces bot controlleurs ont pour charge d'envoyer
les ordres aux différents stations dont ils sont responsables. Il est à noter qu'un
bot peut recevoir les ordres de plusieurs bot controller et cela dans un but de resistance
aux attaques. Si un bot controller tombe mais que les envois d'ordres sont suffisament redondés
au sein du botnet alors la perte d'un bot controller n'est pas handicapante.

\subsection{Risque pour un botnet}


Un botnet peut se faire attaquer. Les différentes parties du botnet représente des cibles et
d'importance différentes.

\subsubsection{Botmaster}

Le coeur d'un bot net est souvent un serveur IRC dans le modèle hierarchique.
Ce serveur est souvent caché et le mettre hors service peut déconnecter l'envoi d'ordre, mais ne peut en revanche
pas soigner les vulnérabilités des différentes stations. Ainsi si le serveur IRC de commandement parvient à se remettre
à fonctionner dans un autre endroit, le botnet est complètement opérationnel.

Cependant, la prise par des autorités compétentes du botnet peut mener à l'exploitation des adresses IP qu'il contient
afin de trouver tous ses membres et pouvoir prendre des mesures comme mettre les adresses IP des noeuds du botnet sur liste noire.
Ainsi, il est clair que prendre ce serveur de commandement est de loin le meilleur remède contre les botnet.

\subsubsection{Bot controller}

Une cible plus modeste pourrait être le bot controller. Ces intermédiaires disposent d'informations intéressantes. Une liste
partielle des membres sous ses ordres. Une liste de taches à envoyer aux bots.

L'arrêt d'un bot controller n'aura souvent pas un effet très fort sur le botnet. En effet, la distribution d'ordre
est très fortement redondée dans un botnet, ainsi la deconnexion d'un seul bot controller n'aura pas un
grand effet sur le botnet.

La prise de contrôle est plus intéressante. Elle permet à l'attaquant de disposer d'une liste d'adresses IP de
bots. L'intérêt est ici plus grand, il est possible de blacklister directement les adresses IP des bots concernés afin
de rendre leurs attaques inefficaces. En outre, si l'attaquant n'est pas très regardant sur les envois du fichiers malveillants
il est possible de se faire passer pour un bot controller normal afin d'extraire du botnet des informations plus intéressantes comme
la localisation de nouveaux bots ou de nouveau bot controller voire même du serveur central.

\subsection{Bot}

Le bot est le travailleur du botnet. Son arrêt ne représente pas un grand intérêt pour lutter contre le botnet.
En outre, la prise de contrôle d'un bot dans un modèle hierarchique n'apporte pas souvent un intérêt très grand.
