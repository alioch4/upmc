\documentclass[a4paper,french,12pt]{memoir}
% L'option draft est vraiment utile pour avoir un document sans 
% erreurs de OverFull UnderFull
\usepackage{xltxtra}
\usepackage{wrapfig}
\usepackage[usenames]{color}
\usepackage{xunicode}
\usepackage{fontspec}
\usepackage{graphicx}
\usepackage{polyglossia}
\usepackage{setspace}
\usepackage{url}
\usepackage{appendix}
\usepackage{numprint}
\usepackage{geometry}
\usepackage[xetex]{hyperref}

% == Metadata =============================================
\title{Fiche de lecture}
\author{Rémy \textsc{Leone}, Marc \textsc{van der Wal}}
\def\keywords{upmc, rad}
\def\subject{Fiche de lecture}

\date{\today}

\hypersetup{
    bookmarks=true, % show bookmarks bar?
    pdftoolbar=true, % show Acrobat’s toolbar?
    pdfmenubar=true, % show Acrobat’s menu?
    pdffitwindow=false, % window fit to page when opened
    pdftitle={Rémy Leone - Marc van der Wal - Fiche de lecture}, % title
    pdfauthor={Rémy \textsc{Leone}, Marc \textsc{van der Wal}}, % author
    pdfcreator={XelaTeX}, % creator of the document
    pdfproducer={Rémy \textsc{Leone}, Marc \textsc{van der Wal}}, % producer of the document
    pdfnewwindow=false, % links in new window
    colorlinks=true, % false: boxed links; true: colored links
    linkcolor=red, % color of internal links
    citecolor=green, % color of links to bibliography
    filecolor=magenta, % color of file links
    urlcolor= blue % color of external links
}

%\geometry{hmargin=1.2in,vmargin=1.2in}

\setdefaultlanguage{french}
\setotherlanguage{english}

\setromanfont[Mapping=tex-text]{Gentium Basic}

\newcommand{\HRule}{\noident\rule{\linewidth}{0.6mm}}

\newcommand{\ieme}{\textsuperscript{e}}
\newcommand{\ier}{\textsuperscript{er}}
\newcommand{\iere}{\textsuperscript{e}}

\renewcommand{\textsc}{\uppercase}
\newcommand{\name}{\textsc}
\newcommand{\engl}[1]{\selectlanguage{english}\textit{#1}\selectlanguage{french}}
\newcommand{\newterm}{\textit}
\newcommand{\soft}{\textit}

%\renewcommand{\glossaryname}{Glossaire}
%\renewcommand*{\begintheglossaryhook}{\begin{description}}
%\renewcommand*{\glossitem}[4]{\item[#1] #2 #3 #4}
%\renewcommand*{\atendtheglossaryhook}{\end{description}}
%\makeglossary


\begin{document}


\makeatletter
\thispagestyle{empty}

\begin{center}
	\noindent \textbf{Ministère de l'Éducation Nationale}
	\vspace{0.8cm}

	\noindent \textbf{Université Pierre et Marie Curie}

	\vspace{4.0cm}

	\noindent \LARGE{\textbf{Fiche de lecture}}

	\vspace{1.0cm}

	\noindent \LARGE{\@title}

	\vspace{5.0cm}

	\noindent \normalsize{\@author}

	\noindent \textbf{Resistance aux attaques distribuées}

	\vspace{2.5cm}

	\noindent \textbf{Responsable de l'enseignement: M. Franck \name{Petit}}

    \vspace{0.7cm}

\end{center}

\makeatother

\pagebreak
\thispagestyle{empty}




\pagebreak

\clearpage

%\onehalfspacing

% ==================================================================
% Resumé
\begin{abstract}

Cette fiche de lecture a pour objectif de résumer les contributions 
de l'article \engl{Honeypot-Aware Advanced Botnet Construction and Maintenance}
ainsi que son influence sur de recents travaux en construction de honeypots résistant aux 
botnets anti-honeypots.

Hashtags : \#honeypots, \#securité, \#botnets

\end{abstract}


% ==================================================================
% Table des matières
\newpage \tableofcontents

% ==================================================================
% REMERCIMENTS

\include{tex/thanks}

% ==================================================================
% AVANT-PROPOS
\section*{Introduction}

Les contraintes énergétiques deviennent aujourd'hui un dimension importante de
la conception d'un système d'information.  Les implications sur les plans
économiques et environementaux sont évidentes.

Dans cet article, plusieurs voies d'améliorations des  \textbf{réseaux cablés}
seront présentées. Ces solutions doivent permettre d'économiser de l'énergie
tout en assurant des performances suffisantes pour les réseaux.


% ==================================================================
% CONCLUSION
\section*{Conclusion}


L'influence des différents mécanismes de modélisation du trafic Internet ont
été étudiés au cours de ces simulations. L'amélioration du délai
d'acheminement des paquets était notre premier objectif, puis nous avons mis en
évidence la corrélation entre le délai et la sporadicité. Le choix des
mécanismes de gestion de file d'attente ont également montré toute leur
pertinence, de même que la taille de la file d'attente.

Si la bonne configuration de toutes ces techniques peuvent améliorer la qualité
de service globale du réseau, elles peuvent impacter très négativement les
performances dans le cas contraire. Ainsi, elle doit être gérée avec une grande
attention.  Les bons paramètres étant fastidieux à trouver, la simulation se
révèle ici être une aide précieuse.


% ==================================================================
% ANNEXES

\end{document}
