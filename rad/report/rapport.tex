\documentclass[a4paper]{sig-alternate}
% L'option draft est vraiment utile pour avoir un document sans 
% erreurs de OverFull UnderFull
\usepackage{xltxtra}
\usepackage{wrapfig}
\usepackage[usenames]{color}
\usepackage{xunicode}
\usepackage{fontspec}
\usepackage{graphicx}
\usepackage{polyglossia}
\usepackage{setspace}
\usepackage{url}
\usepackage{appendix}
\usepackage{numprint}
\usepackage[xetex]{hyperref}

\newcommand{\BibTeX}{{\sc Bib}\TeX}


% == Metadata =============================================
\title{Honeypot-Aware Advanced Botnet Construction and Maintenance}
\author{Rémy Léone, Marc van der Wal}
\def\keywords{upmc, rad}
\def\subject{Fiche de lecture}

\date{\today}

\hypersetup{
    bookmarks=true, % show bookmarks bar?
    pdftoolbar=true, % show Acrobat’s toolbar?
    pdfmenubar=true, % show Acrobat’s menu?
    pdffitwindow=false, % window fit to page when opened
    pdftitle={Rémy Leone - Marc van der Wal - Fiche de lecture}, % title
    pdfauthor={Rémy \textsc{Leone}, Marc \textsc{van der Wal}}, % author
    pdfcreator={XelaTeX}, % creator of the document
    pdfproducer={Rémy \textsc{Leone}, Marc \textsc{van der Wal}}, % producer of the document
    pdfnewwindow=false, % links in new window
    colorlinks=true, % false: boxed links; true: colored links
    linkcolor=red, % color of internal links
    citecolor=green, % color of links to bibliography
    filecolor=magenta, % color of file links
    urlcolor= blue % color of external links
}

\setdefaultlanguage{french}
\setotherlanguage{english}

\setromanfont[Mapping=tex-text]{Gentium Basic}

%\setromanfont[Mapping=tex-text]{Times New Roman}
%
%\newcommand{\HRule}{\noident\rule{\linewidth}{0.6mm}}
%
%\newcommand{\ieme}{\textsuperscript{e}}
%\newcommand{\ier}{\textsuperscript{er}}
%\newcommand{\iere}{\textsuperscript{e}}
%
%\newcommand{\name}{\textsc}
\newcommand{\engl}[1]{\selectlanguage{english}\textit{#1}\selectlanguage{french}}
%\newcommand{\newterm}{\textit}
%\newcommand{\soft}{\textit}

\begin{document}


\maketitle




%\onehalfspacing

% ==================================================================
% Resumé
\begin{abstract}

\end{abstract}


% ==================================================================
% AVANT-PROPOS
\section{Introduction}

\section{Corpus existant}

Il faut mettre ici les travaux des autres gus qui bossent dans le secteur

\section{Détection d'un honeypot}

Les concepteurs de botnets ont plusieurs moyens afin de détecter si
une unité d'un botnet est un honeypot ou bien un zombie digne de confiance.
Dans cette partie nous allons exposer les différentes stratégies que peut employer
un botnet afin de faire le tri entre les unités dignes de confiance et les honeypots.

\subsection{Sonde}

Un outil de prédilection pour trouver des honeypots est une sonde (\textit{probe} en anglais).
Une sonde, est un bot non controlé par un attaquant que le botmaster ou les botcontrollers de confiance
peuvent interroger pour savoir si un message précis à été reçu.

\subsection{Détection par défis d'envoi de contenu malveillant}

En raison de la dimension légale et juridique liée au spam et à l'envoi de fichiers malveillants,
il est clair que les administrateurs système ne peuvent pas contribuer de manière trop zélée au
activités malveillantes du botnet. On tombe ici sur un grave problème, les botnets en testant le
zèle de certains noeuds auront toujours la possibilité de détecter plus ou moins finement les intrus.

De ce fait une stratégie possible pour détecter un honeypot serait de lui envoyer des requêtes comme
``Envoie \numprint{5000} mails contenant des pièces jointes malveillantes aux adresses suivantes''. Si la
station refuse, cela peut signifier qu'elle est momentanément inefficace, ou bien qu'il s'agit d'un honeypot.

Cette stratégie est assez difficile à contrer car l'administrateur du botnet peut
parfaitement mettre en œuvre des sondes parmi les stations (des \textit{probes}, dans
le texte) afin de savoir si une station particulière est de bonne foi
ou non.

Il existe différentes façons d'utiliser cette stratégie.
La première consisterait à utiliser cette stratégie avant de faire rentrer une station dans le botnet.
Il est facile pour un concepteur de botnet de ne donner accès au reste du botnet que si une station
a prouvé son efficacité. Cette méthode a pour avantage de bloquer les honeypots avant qu'ils ne rentrent
dans le botnet.
Une seconde stratégie pourrait consister à remettre en question la confiance que le botnet a dans une station 
de manière périodique. On pourrait placer dans la liste des cibles d'une station une sonde
de confiance aléatoirement choisie et contrôlée par le botnet. Cette stratégie a pour avantage de bloquer les tentatives de prise
de contrôle d'un noeud auxquels le botnet aurait accordé sa confiance.
Ces deux possibilités de vérification sont cumulables.

\subsection{Détection par requête Web}

Une autre façon de détecter si une station est compromise ou non consiste à lui demander d'effectuer
des requêtes HTTP sur une liste de site donnée. Cette procédure est plus difficile à gérer pour un administrateur
sécurité, car il est difficile de savoir si une requête Web est légitime ou bien si elle fait partie d'une tentative
de déni de service distribué. Cette difficulté de faire la différence entre les deux cas rend
la tache de détection des botnets plus difficile.


\subsection{Une lutte difficile}

Il est difficile de savoir si une adresse est un capteur ou non; c'est la raison principale pour laquelle
les honeypots peuvent se faire repérer.
En outre, lorsque une station (honeypot ou non) reçoit une liste de choses à faire, les cibles peuvent
être dissimulés à l'intérieur d'un fichier binaire difficile à analyser afin d'avoir la liste des cibles sans
leur envoyer du contenu. En outre, il est également possible pour un botnet de chiffrer ses communications entre
stations aussi bien que les listes de cibles afin de rendre la rétro-ingénierie plus délicate.

\subsection{Contrôle d'un noeud décideur}

La plupart des administrateurs sécurité hésitent à prendre le contrôle quand ils en ont l'occasion d'un
noeud décideur dans un botnet. Cela pour des raisons juridiques essentiellement. Il est clair que du trafic
partant d'une station est au yeux de la loi la responsable. Ainsi, si une station donneuse d'ordres venait à tomber
sous la responsabilité légale d'un administrateur sécurité, il pourrait être poursuivi pour les dégâts causés.
Ainsi les administrateurs préfèrent souvent éviter de prendre des risques et ne veulent pas prendre le contrôle d'un
noeud donneur d'ordre. Le fait de ne transmettre aucun ordre transforme un bot contrôleur en un ``trou noir'', ou \textit{sinkhole}.
Ce qui le rend bien entendu détectable via un mécanisme de détection à base de sonde comme évoqué précédemment.
Bien que l'utilisation du bot contrôleur pourrait donner de précieuses informations sur le fonctionnement du botnet
(autres stations connectées, liste des tâches en cours, cibles futures, etc.), le risque légal est beaucoup trop
grand et les administrateurs se contentent souvent de mettre sur liste noire les adresses IP répertoriées dans le
bot contrôleur et de ne pas prendre d'autres mesures.

\section{Botnets pair-à-pair}

\subsection{Intérêt du modèle pair-à-pair}

Les modèles actuels de botnet sont majoritairement hiérarchiques. Les
\textit{bot controllers} sont nécessaires afin de faire fonctionner des botnets
de taille importante. Cependant, la configuration d'adresses DNS
dynamique peut demander des ressources temporelles et financières importantes. En
outre, les fournisseurs d'adresses DNS dynamiques peuvent, en partenariat avec
les autorités, trouver les comportements suspects afin de neutraliser des
botnets.

Une parade contre cette faiblesse de passage à l'échelle des botnets
hiérarchiques est l'utilisation de botnets pair à pair, ne reposant plus sur une
hiérarchie mais davantage sur des connexions locales.


\subsection{Un ver en deux temps pour construire un botnet pair à pair}

Afin de construire un réseau pair à pair malveillant, un ver peut être utilisé.
Ce ver fonctionne essentiellement en deux temps :

\begin{description}

\item[Vérification :] Le ver doit s'assurer que la station qui est en cours de
	traitement n'est pas un honeypot. Pour ce faire, il peut utiliser une
	méthodologie similaire à celle présentée précédemment dans l'article. De
	ce fait, si la station n'est pas digne de confiance pour le botnet, alors
	le ver n'essaye pas de faire rentrer la station dans le botnet et aucune
	information ne sera donnée à cette station. Cependant, si la vérification
	aboutit, alors le ver peut passer à la seconde étape.

\item[Enregistrement :] Cette étape a pour but d'envoyer à la station
	fraîchement compromise de nouvelles cibles à infecter avec le même ver
	qui a été utilisé pour la compromettre elle-même. Les listes en question
	sont nommées des \textit{buddy lists}. L'avantage énorme de ces buddy
	lists réside dans le fait qu'en cas de prise de contrôle d'un nœud du
	botnet par un attaquant, il est difficile d'avoir accès à plus de nœuds
	que ceux de la buddy list. De cette façon, le réseau résiste aux
	attaques. En outre, même si des nœuds sont attaqués et mis hors
	service par des attaquants, l'administrateur du botnet peut toujours
	donner des ordres via les noeuds encore connectés.

\end{description}

Ce ver fonctionnant en deux temps facilite significativement la création de
botnets, tout en les rendant assez résistantes aux attaques.

\section{Travaux ultérieurs}

Un modèle pair à pair n'est pas nécessairement le plus adapté pour
gérer un botnet. Cependant un modèle hybride permet de combiner les
avantages du modèle hierarchique et du modèle distribué. Cette partie
est une ouverture vers les travaux plus récents qui porte l'influence
des idées de l'article de Zou et Cunningham \cite{main}.

\subsection{Présentation du modèle hybride}

Le modèle pair à pair est très flexible. Il existe de nombreuses différences
entre les modèles utilisées par Napster, BitTorrent, ou d'autres protocoles.
L'un des intérêts d'utiliser du modèle pair à pair et d'augmenter la robustesse
du botnet en cas d'attaque. Il est clair que tous les modèles de pair à pair ne
peuvent pas convenir. Un modèle pair à pair centralisé, tel que celui utilisé
par Napster, ne pourrait pas fonctionner car il suffirait comme dans le cas
distribué de neutraliser le serveur principal pour arrêter le botnet.

\subsection{Amélioration du mécanisme de construction du botnet}

L'une des améliorations des travaux ultérieurs est le mécanisme de réinfection \cite{p2p}.
Au moment de la construction d'un botnet, un noeud A peut choisir de re-contaminer un noeud B
d'ores et déjà intégré au botnet. Pour ce faire A envoie un B une liste d'hôte qu'il doit intégré
dans sa liste de pair de façon aléatoire. De cette manière, dans l'hypothèse où un
noeud serait capturé; il serait très difficile pour un attaquant du botnet de retrouver
l'historique d'infection d'un noeud particulier. Cette amélioration rends le botnet
plus difficile à mesurer pour ses attaquants.


\subsection{Nouvelles parades}

La robustesse des botnet hybride se base sur le fait que les listes de pairs se mettent à jour
régulièrement et se propage rapidement. En laissant un
honeypot se faire infecter puis envoyer à un maximum d'honeypot ce bot, il est possible d'attaquer
le botnet en faisant croire au controlleur du botnet que les honeypots peuvent être des servants
et récupérer des adresses IP d'hôte infecté. Cette stratégie est d'autant plus efficace si les
attaquants du botnet disposent de grandes plages d'adresses IP.


\subsection{Ouvertures}

Le modèle pair à pair prendra plus d'importance dans les années à venir en raison
de ses nombreux avantages. La robustesse de ce modèle le rends particulièrement résistant aux attaques
classiques. Une parade face à ce genre de botnet peut être trouvée via l'utilisation des honeypots afin
de trouver des informations pertinentes vis à vis de ces systèmes. L'étude des honeypot doit devenir
un domaine d'étude de premier plan pour lutter contre les botnets.



% ==================================================================
% CONCLUSION
\chapter*{Conclusion générale}
\addcontentsline{toc}{chapter}{Conclusion}


Nous avons vu dans ce papier quelles étaient les techniques pour construire
des botnets resistants contre les honeypot. Le constat est assez clair, si
les administrateurs systèmes ne prennent pas le contrôle de noeud du botnet,
ils ne disposent de peu d'informations et de peu de moyens pour agir efficacement
et durablement pour la neutralisation des botnets.

Cet article sort également de sa dimension strictement technique pour toucher
un problème de fond de responsabilité judiciaire des administrateurs. Tant
que les administrateurs de parc informatique n'auront pas les mains plus libres
pour essayer de rentrer dans le botnet quitte à envoyer des flux malveillants, les
controlleurs de botnet auront toujours une longueur d'avance et les administrateurs
seront toujours condamnés à avoir une defense non offensive contre les botnet. Une
institution non gouvernementale et internationale pourrait permettre d'encadrer ces
cas houleux afin d'avoir à terme une methodologie efficace contre les botnets qui seront
toujours libres d'agir à l'encontre de tout impératif légaux.



% ==================================================================
% BIBLIOGRAPHIE

\bibliographystyle{abbrv}
\bibliography{rapport.bib}

% ==================================================================
% ANNEXES

\end{document}
