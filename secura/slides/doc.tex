\begin{document}

\maketitle

\section{Introduction \& Définition}

\subsection{Introduction}
\begin{frame}{Introduction}{Protection des données}

\begin{block}{Mécanisme de protection des données}
\begin{itemize}
\item Consentement des utilisateurs
\item Charte de vie privée
\item Contrôle d'accès
\end{itemize}
\end{block}

\begin{block}{Problèmes}
\begin{itemize}
\item Peu d'utilisateurs lisent les chartes
\item Modifications des chartes
\end{itemize}
\end{block}

\end{frame}

\subsection{Modèles}

\begin{frame}{Modèles \& Menaces}

\begin{block}{Modèles}
\begin{itemize}
\item Intégrité
\item Disponibilité
\item Confidentialité \& Intimité
\end{itemize}
\end{block}

\begin{block}{Attaques}
\begin{itemize}
\item Vol d'identité
\item Usurpation
\item Profils illégitimes
\end{itemize}
\end{block}
\end{frame}


\begin{frame}{Modèles \& Menaces}
\begin{block}{Garantir}
\begin{itemize}
\item Anonymat des interactions
\item Inobservabilité
\item Non recoupement d'informations
\item Intraçabilité
\end{itemize}
\end{block}
	
\end{frame}
\section{Réseaux centralisés}

\subsection{Facebook}
\begin{frame}{Facebook}

\begin{block}{Caractéristiques principales}
\begin{itemize}
\item Relation symétrique
\item Contrôle d'accès ``potentiel'' sur chaque objet
\item Applications tierces
\end{itemize}
\end{block}

\begin{block}{Atouts}
\begin{itemize}
\item Grande communauté
\item Très réactifs face aux améliorations des concurrents
\item Suggestions de listes
\end{itemize}
\end{block}

\end{frame}

\subsection{Google+}
\begin{frame}{Google+}

\begin{block}{Caractéristiques principales}
\begin{itemize}
\item Relation asymétrique
\item Contrôle d'accès sur chaque objet
\item Liste de diffusion ``vissées'' à l'interface graphique
\end{itemize}
\end{block}


\begin{block}{Atouts}
\begin{itemize}
\item Unification des différents services Google
\item Sécurisé par ``défaut''
\end{itemize}
\end{block}

\end{frame}

\subsection{Linkedin}
\begin{frame}{Linkedin}
\begin{block}{Caractéristiques principales}
\begin{itemize}
\item Relation symétriques
\item Contrôle d'accès sur tous les objets
\item Milieu soucieux des conditions d'utilisation
\end{itemize}
\end{block}

\begin{block}{Atouts}
\begin{itemize}
\item Contrôle très étroit sur les applications tierces
\item Certification TRUSTe
\end{itemize}
\end{block}

\end{frame}

\section{Réseaux décentralisés}

\subsection{Vis a vis}

\begin{frame}{Réseaux décentralisés}{Vis-à-vis}
\begin{block}{Vis-à-vis}
\begin{itemize}
\item Affecter un VIS (Virtual Individual Server) à chaque utilisateur
\item Auto organisation en overlay
\item Hébergeur : Cloud (ex.: Amazon EC2\ldots)
\end{itemize}
\end{block}

\begin{block}{Problèmes}
\begin{itemize}
\item Souvent payant
\item La confiance est simplement déplacée vers l'hébergeur
\end{itemize}
\end{block}
\end{frame}

\subsection{Safebook}

\begin{frame}{Réseaux décentralisés}{Safebook}
\begin{block}{Safebook}
\begin{itemize}
\item Relations basées sur celle de la vie réelle
\item Présentation d'une pièce d'identité pour joindre le réseau
\item Les noeuds sont mirrorés
\end{itemize}
\end{block}

\begin{block}{Problèmes}
\begin{itemize}
\item Disponibilité
\item Synchronisation des données
\end{itemize}
\end{block}
\end{frame}

\section{Attaques}

\begin{frame}{Attaques}
    Trois catégories:
    \begin{itemize}
        \item Obtention de l'accès au compte de la victime
            \begin{itemize}
                \item Exemple: Ingénierie sociale classique
            \end{itemize}
        \item Fuite d'informations privées vers l'extérieur
            \begin{itemize}
                \item Exemples: Ingénierie sociale inversée, infiltration par
                    botnet
            \end{itemize}
        \item Attaques sur l'anonymat
            \begin{itemize}
                \item Exemple: Désanonymisation par vol d'historique
            \end{itemize}
    \end{itemize}
\end{frame}

\subsection{Fuites de données}
\begin{frame}{Fuites de données}
\begin{itemize}
\item 1\% des fuites de données sont dues à des attaques externes
\item Utilisation des réseaux sociaux dans les entreprises
\end{itemize}
\begin{block}{Problèmes}
\begin{itemize}
\item Dissémination des données personnelles
\item Fuites de données
\item Virus \& Logiciels malveillants
\end{itemize}
\end{block}
\end{frame}

\subsection{Ingénierie sociale inverse}
\begin{frame}{Ingénierie sociale inverse}
    \begin{itemize}
        \item Au lieu que l'attaquant contacte la victime, il s'arrange pour
            que la victime le contacte en premier
        \item Trois méthodes:
            \begin{itemize}
                \item Apparaître dans les recommendations d'utilisateurs à
                    ajouter comme ami (ex.: Facebook)
                \item Apparaître dans la liste des derniers visiteurs (peu
                    efficace seul)
                \item Mettre en place un profil racoleur (même âge, même
                    localité), surtout sur des réseaux orientés ``rencontres''.
            \end{itemize}
        \item L'attaquant s'arrange pour que la victime l'ajoute dans ses amis
    \end{itemize}
\end{frame}


\subsection{Désanonymisation}
\begin{frame}{Désanonymisation}
    \begin{itemize}
        \item Attaque par vol d'historique
            \begin{itemize}
                \item Mise en place d'une page Web truquée avec liste de liens
                \item Liens déjà visités: affichés avec une image de fond
                    ``piégée''
                \item L'attaquant connaît ainsi les sites déjà visités par la
                    victime  \pause
                \item Exemple: \url{http://didyouwatchporn.com}
            \end{itemize} \pause
        \item Idée générale: appliquer cette attaque aux pages spécifiques à un
            groupe
            \begin{itemize}
                \item Identification des groupes auxquels appartient un
                    utilisateur
                \item Identification unique parmi tous ses homonymes
            \end{itemize}
    \end{itemize}
\end{frame}

\subsection{Autres méthodes}
\begin{frame}{Autres méthodes}
    \begin{block}{Infiltration par botnet}
        \begin{itemize}
            \item Utiliser des botnets pour créer des comptes malveillants avec de faux
profils aguicheurs
            \item Les victimes les ajoutent comme amis
            \item Les informations uniquement visibles aux amis peuvent ainsi fuiter
        \end{itemize}
    \end{block}
    \begin{block}{Compromission d'applications tierces}
        \begin{itemize}
            \item Si les applications tierces stockent les données des
utilisateurs, une
                vulnérabilité dans l'application et là, c'est le drame.
        \end{itemize}
    \end{block}
\end{frame}

\section{Ouvertures \& Conclusion}

\subsection{Ouvertures}
\begin{frame}{Scénarios possibles}
    \begin{itemize}
        \item Mise en place d'autorités comme TRUSTe
        \item Comment assurer un contrôle des identités?
        \item Comment assurer l'interopérabilité entre les réseaux sociaux ?
    \end{itemize}
\end{frame}
\subsection{Conclusion}
\begin{frame}{Conclusion}
    \begin{itemize}
        \item Réseaux sociaux: prédominants, et pourtant négligés en termes de
            sécurité
        \item Contrôle d'accès le plus intéressant: DAC
        \item Qui a la propriété et les droits d'auteur sur les contenus
            des utilisateurs?
        \item Disponibilité: critère important, paradoxalement optimal dans des
            réseaux sociaux \textbf{centralisés}
    \end{itemize}
\end{frame}
\end{document}

% vi:ts=4:et:sw=4:tw=80:
