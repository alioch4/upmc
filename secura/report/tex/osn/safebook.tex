\subsection{Safebook}

\begin{flushright}\textit{Auteur: Riad Mazloum}\end{flushright}

Safebook est un réseau social pair à pair qui essaie de faire face aux
problèmes d’intimité et de confiance. Le stockage et le routage des
informations sont faits par des nœuds qui se font mutuellement confiance, ce
qui veut dire que ces nœuds se basent sur de relations établies dans la vie
réelle. Pour mieux comprendre l’architecture de Safebook, il faut d’abord
définir les niveaux de services des réseaux sociaux:

\begin{itemize}
\item Social Network (SN), la représentation numérique des membres et les
relations entre eux;

\item Social Networking Service (SNS), le niveau de l’application qui est géré
par le fournisseur du SNS;

\item Communication and Transport (CT), les services de communication et
transport fournis par les réseaux.
\end{itemize}

Safebook implémente ces trois niveaux :

\begin{itemize}
\item SN : le niveau implémenté par la couche de réseau centrée sur
l’utilisateur ;

\item SNS : l’implémentation pair-à-pair des services ;

\item CT : ce niveau est implémenté par l’Internet. On peut distinguer aussi un
autre service, « Trusted Identification Service (TIS) », qui attribue à chaque
nœud un identifiant unique et un pseudonyme. 
\end{itemize}

Par exemple, pour que $u$ puisse rejoindre le réseau, il doit être invité par $v$, un
membre du réseau. L'utilisateur $u$ doit contacter le TIS et fournir une paire de clefs. Le TIS
vérifie l’identité de $u$ et lui attribue un certificat. L’intimité est fondée sur
un mécanisme de réseau de confiance. Tous les nœuds dans un overlay n’ont
conscience que de leurs voisins directs. L’utilisateur attribue un niveau de
confiance spécifié. Ainsi les données peuvent être classifiées en trois
niveaux: privé, protégé et public. Les données privées ne sont pas
publiées, les données protégées sont publiées et chiffrées, et enfin, les publiques sont
publiées sans être chiffrées. Chaque nœud possède des \textit{nœuds miroirs}, ou \textit{nœuds
voisins}, où le nœud peut stocker ses données, qui sont chiffrées.  La source
des requêtes est toujours cachée, car les requêtes sont propagées
récursivement.  Grâce à ces propriétés, Safebook répond aux problèmes d'intimité
des données des membres. On peut résumer le modèle de sécurité de Safebook de la
manière suivante :

\begin{itemize}
\item Contrôle d’accès : chaque membre décide ce qu’il veut partager et avec
qui, par les relations de confiance entre les membres;

\item Anonymat d’interaction : toutes les interactions sont basées sur le
concept de confiance, et tous les autres membres avec lesquels le membre ne
partage rien n’ont pas d’accès aux informations non privilégiées, les données
étant chiffrées;

\item Applications tierces : la notion n'existe pas;

\item Fuites de données : toutes les opérations sont contrôlées par le membre, et
aucun membre non privilégié ne peut obtenir d’accès aux données privées;

\item Synchronisation des données : un problème de synchronisation est encore présent;

\item Disponibilité : dépend du nombre de nœuds "miroirs".
\end{itemize}
