\subsection{Linkedin}

\begin{flushright}\textit{Auteur: Rémy Léone}\end{flushright}

Fondé en 2002 et lancé en 2003, présent dans 200 pays, Linkedin possède plus de
150 millions d'utilisateurs actifs. \cite{linkedin-stats} Linkedin est un réseau
social orienté vers le monde de l'entreprise. Les informations extraites des
profils des utilisateurs concernent leurs informations professionnelles telles
que leurs formations, leurs emplois et les recommandations de leurs employeurs.

% -Application tierces

Des applications tierces sont disponibles pour Linkedin. Leur principale
utilité consiste à afficher sur le profil Linkedin des informations en
provenance d'autres services Web. Ainsi les informations de ces applications
tierces vont vers Linkedin mais n'en sortent pas.

% -Fuite de données

Les données extraites sur les profils des utilisateurs sont utilisées par
Linkedin afin de publier des annonces liées avec les centres d'intérêts et les
attentes des utilisateurs.

% -Anonymat des interactions
% -Contrôle des données utilisateurs

Les critères de visibilité d'un profil utilisateur sont paramétrables. Il est
possible de restreindre l'accès à une certaine classe d'informations (par
exemple les employeurs ou bien les recommandations) qu'à une certaine classe
d'utilisateurs connectés. En outre, les informations liées au profil peuvent
être visibles publiquement ou totalement privées.

Il est à noter que Linkedin est certifié TRUSTe \cite{truste}, ce qui, dans
le contexte des menaces contre la vie privée des utilisateurs de la part des
réseaux sociaux, donne à Linkedin un cachet auprès de l'entreprise où les
contraintes sur les informations diffusées sont fortes.

\subsubsection{TRUSTe}

TRUSTe est un groupe indépendant à but non lucratif. \cite{truste} Son but est
de faire des audits sur la protection des informations personnelles des
utilisateurs, ainsi que sur les politiques de confidentialité mises en œuvre
par les entreprises vis-à-vis de leurs clients.

Les permis que TRUSTe a délivré à Linkedin ont plusieurs objectifs :

\begin{description}

\item[Conformité aux lois :] Les lois fédérales américaines et les lois
européennes collaborent afin de mettre en place des cadres communs de
législation et faciliter les coopérations légales en cas de litige. TRUSTe
permet de valider la confirmité d'un site avec ces lois relatives à la vie
privée à un niveau international;

\item[Notifications des utilisateurs :] Tous les utilisateurs peuvent avoir un
contrôle très fin sur l'ensemble des informations qui les concernent et la
façon dont ces informations sont utilisées. En particulier, à chaque changement,
ils sont informés et ont la possibilité de retirer leurs informations si des
changements de politique ne sont pas conformes à leurs souhaits. Le
consentement des utilisateurs à l'utilisation de leurs données personnelles est
un critère absolument central au sein de l'évaluation menée par TRUSTe;

\item[Audit réguliers :] TRUSTe se réserve le droit d'auditer quand bon lui
semble des entreprises ou organisations détentrices des permis qui ont été
précedemment remises.

\end{description}
