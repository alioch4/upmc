\subsection{Vis-à-Vis}

\begin{flushright}\textit{Auteur: Patrick Raad}\end{flushright}

Vis-à-Vis est une plate-forme alternative permettant de résoudre les problèmes
d’intimité qu’on peut trouver dans les OSN centralisés. \cite{vis-a-vis} Chaque
utilisateur possède une machine virtuelle, hébergée par exemple sur Amazon EC2,
dans laquelle il va stocker ses informations personnelles. L’utilisateur
conserve le contrôle et l’accès à ses informations, sans être obligé d’accepter une
charte du site hébergeur. Afin de permettre le partage de son profil, chaque
VIS (Virtual Individual Server) va s’auto-organiser en réseaux couvrants 
(«overlay networks»), où chaque overlay est un groupe avec lequel l’utilisateur
veut partager ses données. Bien qu’un VIS reste vulnérable aux attaques
classiques (DoS, sybil attacks, etc.), il appartient à un utilisateur, donc il est
moins concerné par les attaques qu’un data center qui possède des millions
d’utilisateurs. En outre, l’utilisation d’un cloud tel que Amazon EC2 assure
des garanties de disponibilités et de contrôle d’accès au données physiques, qui
ne pourrait pas être obtenues facilement avec de l’auto-hébergement. De plus,
un individu possédant un VIS peut aussi contrôler la précision de ses
coordonnées (« locations »). Il peut mentir à propos de ses coordonnées
actuelles, comme il peut les publier avec une précision différente pour chaque
groupe pour lequel il appartient. Vis est un système payant, et ainsi, son
utilisation au détriment de réseaux sociaux centralisés et gratuit ne se fera
pas facilement.
