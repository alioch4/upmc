\subsection{Facebook}

\begin{flushright}\textit{Auteur: Rémy Léone}\end{flushright}

Facebook est de loin le réseau social le plus actif de la planète et possède
plus de 845 millions de membres actifs. \cite{facebook-people}

% -Contrôle des données utilisateurs

Facebook a pendant été longtemps critiqué pour son aspect ``Big Brother'' et
ses politiques de sécurité des utilisateurs mal définies. \cite{facebook-eff}
Bien que de grands progrès soient constatables, il faut cependant rester
vigilant vis-à-vis du cycle de vie des données personnelles de l'utilisateur, et
en particulier de leur destruction.

En effet, au moment où un utilisateur souhaite supprimer son profil et effacer
les traces de ses activités passées des serveurs de Facebook, cela est rendu
impossible par Facebook qui se contente de garder les données et de les rendre
inaccessibles pour les autres utilisateurs. Cependant, les données restent
encore utilisables par Facebook, qui en garde le contrôle total.
\cite{facebook-delete-data}

En ce qui concerne la gestion des données utilisateur en tant que telles, des
listes de diffusion de contenu sont utilisables et ont été considérablement
améliorées. Ainsi, un utilisateur a la possibilité de contrôler avec précision
les personnes qui ont accès à chacun des contenus qu'il poste. Ainsi, le
contrôle d'accès est efficace, bien qu'aucun moyen simple ne soit disponible
pour supprimer d'un seul coup l'ensemble des données d'un
utilisateur. \cite{facebook-lists}

En outre, il est à noter qu'il est possible pour un utilisateur de télécharger
une copie de toutes les données que Facebook possède à son sujet. Cette archive
n'est pas disponible immédiatement mais Facebook peut la fabriquer pour
quiconque en faisant la demande.

% -Application tierces

Facebook a lancé sa plate-forme d’applications en 2007.
\cite{facebook-app-privacy} Son but était d’améliorer l’expérience sociale
entre les différents utilisateurs. Facebook a permis aux applications tierces
d'extraire des informations personnelles d’un individu comme sa liste de
contact ou l'accès à ses photos ou vidéos. En réalité, peu d’applications
nécessitent l’accès à une telle variété de données pour fonctionner. Les
applications ont besoin d'autorisations de la part des utilisateurs afin
qu'elles puissent fonctionner, cependant, malgré les indications explicites qui
apparaissent au moment où un utilisateur veut installer une application,
certaines d'entre elles sont encore non lues par les utilisateurs, qui exposent
ainsi des applications à accéder pour des raisons floues à toutes leurs données,
les exposant ainsi à de plus grands risques de sécurité et notamment de
confidentialité.

Cette situation est similaire à celle d'une personne qui déciderait de joindre
une association dans laquelle on lui demanderait sa date de naissance, son
adresse, sa religion et ses contacts proches alors que seul les deux premiers
étaient vraiment pertinents.

Les problématiques concernant l'intimité dans Facebook sont multiples.
\cite{facebook-user-privacy} La politique de partage des données des
applications tierces est presque invisible aux yeux de l’utilisateur moyen.
L’utilisateur est informé que ses informations personnelles et les informations
de ses amis dans sa liste de contact vont être partagées au moment de
l'installation d'une application tierce. La plupart du temps ce message n’est
pas détaillé, et les utilisateurs l’ignorent complètement.  Moins de 2\% des
utilisateurs font attention au contrat de licence avant d’installer une
application, donc dans ce cas on peut dire qu’il y a une négligence de la part
des utilisateurs des réseaux sociaux. Un autre point qui est assez important,
c’est que les développeurs sont anonymes, donc personne ne peut garantir qu’il
n’y ait pas de fuite de données au delà des limites définies par Facebook.

% -Fuite de données

% -Anonymat des interactions
