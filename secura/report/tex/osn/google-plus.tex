\subsection{Google+}

\begin{flushright}\textit{Auteur: Rémy Léone}\end{flushright}

Google+ est une réseau social jeune lancé le 28 juin 2011 par Google.
\cite{google-plus-annonce}

Google+ possède une interface graphique et un fonctionnement très différent de
celui de son principal rival, Facebook.

Tout d'abord, les connexions dans Google+ sont asymétriques: il n'est pas
nécessaire d'être ``ami'' avec une personne pour accéder à un sous-ensemble de
ces informations.  Il est à noter que Facebook a réagi à cette fonctionnalité
en ajoutant des possibilités d'abonnements à des flux d'informations concernant
un utilisateur au sein de Facebook. \cite{facebook-subscribe}

Les contacts rentrent dans des cercles. \cite{google+-circles} Ce sont ces
cercles qui vont également faire office de système de contrôle d'accès. En
effet, il est possible pour un utilisateur de ne publier que certains contenus
sur un ensemble de cercles précis. Les utilisateurs ont la possibilité au
moment où ils ajoutent des informations au sein de Google+ de choisir les
cercles avec qui ils ont envie de la partager. Ainsi, il est possible de
partager une information, telle qu'une photo ou une publication, de manière
publique (visible pour toute personne consultant le profil de l'utilisateur) ou
bien de manière restreinte via les cercles. Ainsi, l'utilisateur a un plus
grand contrôle sur les personnes qui peuvent consulter ses données. En outre,
et contrairement à Facebook, ce système de cercle est véritablement vissé à
l'interface de l'utilisateur, et il est impossible pour un utilisateur de ne pas
s'en servir. À l'inverse, il est totalement possible pour un utilisateur au sein
de Facebook de ne pas utiliser de listes.  Il est à noter que Facebook tente
de faire des suggestions sur l'utilisation de
listes. \cite{facebook-smart-lists} L'exemple de tous les collègues ou bien de
tous les membres de la famille rassemblés dans une liste respective est une
suggestion qui apparait être une réaction aux système de Google+.

% -Application tierces

Des applications tierces telles que des jeux peuvent être utilisées sur
Google+.  Cependant, pour le moment, Google n'ouvre pas véritablement ce
service à n'importe quel utilisateur. Bien que l'API soit publique, le fait
d'arriver jusqu'au yeux de l'utilisateur moyen n'est pas pour le moment pas à
la portée d'un utilisateur qui n'est pas clairement identifié par Google. Ainsi
les attaques par applications tierces ne sont pas d'actualité.
\cite{google-plus-third}

% -Fuite de données

À ce jour, aucune fuite de données systématique n'a été constatée au sein de
Google+. Bien que par le passé, des problèmes ont été remarqués au sein de
Google Buzz, un ancien réseau social également conçu par Google
\cite{buzz-fail}, les modèles d'accès aux informations via les cercles font
qu'il est très difficile de partager des informations avec des personnes non
autorisées sans le consentement explicite de l'utilisateur.

% -Anonymat des interactions
% -Contrôle des données utilisateurs
% - Remarque

Il est à noter que bien que l'objectif principal de Facebook et de Google+ soit
de mettre en contact leurs utilisateurs, un de leurs moyens de financement est
la publicité envoyée aux utilisateurs en fonction des données qu'ils possèdent.
Les utilisateurs restent sur Facebook en moyenne plus longtemps que sur
Google+, ce qui pourrait sembler être une faiblesse pour Google+. Il n'en est
rien, car depuis que les politiques de confidentialité des données sur les
différents services de Google ont été unifiées \cite{spyw}, l'utilisateur
reçoit des publicités ciblées par Google à chaque fois qu'il utilise un de
leurs services. Ainsi, peu importe qu'il reste que quelques minutes sur Google+
au lieu de quelques heures sur Facebook, Google recoupe les informations de
tous ses services afin d'envoyer de la publicité et du contenu de manière
ciblée et sur l'ensemble de ses services.
