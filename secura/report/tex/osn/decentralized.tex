\subsection{Réseaux décentralisés}

\begin{flushright} \textit{Auteur: Rémy Léone} \end{flushright}

De nombreuses critiques sont émises à l'encontre des réseaux sociaux.  En
particulier, les administrateurs de tels systèmes sont accusés de vendre les
informations personnelles des utilisateurs au plus offrant pour effectuer des
campagnes publicitaires.

Ainsi, des projets orientés vers la sécurité et la confidentialité des données
utilisateurs ont été lancés afin de créer des réseaux sociaux décentralisés et
soucieux de la protection des données individuelles.

\subsubsection{Architecture}

Les architectures des différents réseaux sociaux décentralisés sont basés sur
des idées similaires. Les données sont disponibles sur un certain ensemble de
noeuds, les connexions se font via des concentrateurs ``hub'' qui permettent
aux utilisateurs de se retrouver entre eux. \cite{safebook, vis-a-vis}

\subsubsection{Modèles de sécurité}

Le modèle avancé par les réseaux sociaux décentralisés est orienté vers le
contrôle total des informations des utilisateurs par ces derniers. Aucune
autorité centrale ne doit pouvoir accéder aux données, même afin d'assurer leur
disponibilité.

Le modèle de contrôle d'accès est proche de celui des réseaux sociaux
centralisés.  Chaque objet peut comporter une liste de contrôle d'accès qui
réglemente finement l'ensemble des interactions que les autres utilisateurs
enregistrés ou anonymes peuvent effectuer avec les données.

Les connexions entre utilisateurs se font par le biais de concentrateurs (``hubs'')
qui permettent aux différents utilisateurs de se connecter les uns aux autres.
Cependant, ce modèle présente des failles dans le cas où le concentrateur est
corrompu ou est indisponible.

\subsubsection{Critiques}

Les réseaux décentralisés sont soumis à de nombreuses critiques.  L'une des
principales critiques est la difficulté de trouver des solutions techniques
testées, maintenues et soumises à des audits réguliers et poussés.  En effet,
les solutions de réseaux sociaux décentralisés s'adressent à un public
d'utilisateurs soucieux de la sécurité et pour qui ces exigences de sécurité
sont particulièrement importantes. Les différents
réseaux sociaux pair à pair ont pour même objectif la distribution de
l’information, au lieu de la centraliser dans une seule autorité tierce.  Cette
distribution se traduit par des pertes de performance et de flexibilité au
niveau du partage des données.

Nous pouvons également évoquer le problème de la disponibilité des données. De
telles solutions exigent qu'un utilisateur garde une station allumée en
permanence pour pouvoir partager tout ou partie de son profil avec d'autres
utilisateurs, surtout s'il en est l'unique possesseur.  S'il éteint la machine
en question, son profil et ses données ne seront plus accessibles par les
autres utilisateurs. Dans un contexte de réseau social, cette indisponibilité
est particulièrement handicapante.

Un autre problème des réseaux décentralisés est le fait de ne pas disposer
d'autorité centrale. Les personnalités politiques et entreprises souhaitent
s'inscrire sur les réseaux sociaux afin d'avoir accès à un plus large public.
Cependant, il n'est pas rare que de faux profils soient en activité afin
d'usurper les identités de noms prestigieux. Ainsi, dans le cas d'un réseau
complètement décentralisé, et sans autorité centrale, il est très difficile pour
un utilisateur légitime de dénoncer des profils mensongers.

Enfin, afin d'être attrayant, un réseau social a besoin d'une ``masse
critique'' d'utilisateurs. Or, l'investissement temporel dans de telles
installations est trop important pour envisager une adoption rapide de la part
d'une communauté de non-initiés.
