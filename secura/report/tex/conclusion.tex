\section*{Conclusion}

\begin{flushright}\textit{Auteur: Rémy Léone}\end{flushright}

Les réseaux sociaux sont aujourd'hui des acteurs lourds du paysage de
l'Internet. La masse des informations personnelles que ces sites gèrent est
devenu importante et malgré cela, les règles de sécurité les plus élémentaires
sont souvent ignorées ou négligées par la part des utilisateurs.

À travers les exemples de réseaux sociaux et leurs modèles de sécurité qui ont
été présentés dans cet article, les évolutions ont pu être présentées et
on peut conclure qu'un contrôle d'accès discrétionnaire est le modèle le
plus intéressant pour les utilisateurs afin d'avoir un contrôle fin sur les
informations qu'ils diffusent.

La propriété des informations que les utilisateurs partagent sur les réseaux
sociaux est également un critère important pour déterminer si le droit à
l'oubli et les autres recommandations des organisations de protection de la vie
privée sont mis en place.

Enfin, la disponibilité des informations est également un critère important afin
de juger de la fiabilité d'un réseau social. Cette disponibilité est optimale
dans le cas d'un réseau social centralisé disposant de vastes ressources, et
pouvant ainsi assurer une redondance importante des données des utilisateurs.
