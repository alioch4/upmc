\section{Attaques}

\begin{flushright}\textit{Auteur: Marc van der Wal}\end{flushright}

Les attaques sur les réseaux sociaux peuvent être divisées en trois catégories.
Premièrement, nous avons les attaques classiques visant à obtenir l'accès au
compte de la victime.  Deuxièmement, d'autres attaques visent à dévoiler des
informations privées sur la victime vers l'extérieur, sans pour autant obtenir
ses identifiants.  Troisièmement, on peut ajouter les attaques sur l'anonymat;
en effet, même si la victime utilise un pseudonyme, la quantité d'informations
recueillie sur elle permet parfois de la confondre.

\subsection{Ingénierie sociale inverse}

Bien que les attaques dites d'\emph{ingénierie sociale} sont des attaques
classiques, connues et polyvalentes, l'\textit{ingénierie sociale inversée},
mise en évidence par Irani \textit{et al.} \cite{dan11}, suit un principe
légèrement différent.

En effet, contrairement à une attaque par ingénierie sociale où l'attaquant
contacte la victime, il est aussi possible de faire en sorte que la victime
contacte l'attaquant.  Cette attaque est rendue possible par trois
fonctionnalités que les réseaux sociaux tendent à mettre en place.

Premièrement, certains réseaux comme Facebook mettent en œuvre une liste de
recommandations d'utilisateurs avec qui l'utilisateur peut nouer une relation
d'amitié: par exemple, deux utilisateurs $A$ et $B$ ont $n$ amis en commun, donc
$B$ est présenté à $A$ comme un ami potentiel.  D'autres heuristiques peuvent
être utilisées, mais celles-ci peuvent être exploitées par un attaquant pour que
la victime l'ajoute dans sa liste d'amis.

Deuxièmement, l'attaque peut également être réalisée par traçage de visites:
certains réseaux sociaux envoient une notification lorsque le profil de
l'utilisateur est visité par une autre personne, et gardent une trace des $n$
derniers visiteurs.  Il suffit que l'attaquant visite le profil de la victime
pour espérer que la victime l'ajoute à ses amis.  Cependant, cette attaque est
généralement peu efficace lorsqu'utilisée seule.

Troisièmement, la mise en place d'un faux profil suffisamment convaincant d'une
personne habitant la même localité et ayant la même âge que la victime permet,
lui, d'effectuer une attaque dite démographique.  Cette dernière attaque est
d'autant plus efficace sur les réseaux sociaux orientés ``dating'', comme Badoo.


\subsection{Désanonymisation}

Wondracek et al. ont avancé la possibilité d'une attaque de désanonymisation en
exploitant l'appartenance de la victime à des groupes. \cite{whkk10}

Cette méthode utilise une attaque par vol d'historique afin de connaître les
pages de groupes visitées par la victime -- ou, du moins, des pages spécifiques à
un groupe particulier et qui ne sont accessibles qu'aux membres. En effet, il est
possible de déterminer l'appartenance à un groupe à partir des URL se trouvant dans un
historique Web.  L'attaquant exploite ensuite le fait que les navigateurs
colorent différemment les liens déjà visités par l'utilisateur. Il construit
donc une page contenant des liens vers chacune de ces URLs, et utilise une
feuille de style CSS spécifique pour qu'une image de fond ``piégée'' ne se
charge que si une URL est visitée.  Ainsi, l'attaquant connaît les pages déjà
visitées parmi les liens de la page, et donc les groupes auxquels la victime
appartient.

Les informations d'appartenance au groupe obtenues de la victime est ensuite
suffisante pour l'identifier de manière unique, ou dans le pire cas, réduire de
manière significative le nombre de personnes pouvant lui correspondre.

\subsection{Infiltration par botnet}

\begin{flushright}\textit{Auteurs: Rémy Léone, Marc van der Wal}\end{flushright}

De nombreux utilisateurs se soucient de leur vie privée et veulent protéger les
informations qui les concernent. Cependant, ils veulent que leurs amis proches
puissent accéder facilement aux informations qui les concernent. 

Malgré les mesures déjà mises en place par certains réseaux sociaux afin de
mitiger le problème de la création automatisée de faux comptes (captchas, etc.),
Boshmaf et al.\ ont montré qu'il reste toujours possible d'infiltrer
un réseau social à l'aide de robots, et d'obtenir des informations privées
d'utilisateurs peu vigilants en les incitant à ajouter ces robots comme amis.
\cite{Boshmaf:264} Ces robots utilisent des profils fictifs et souvent
aguicheurs; le but étant toujours de faire fuiter des informations
privées vers l'extérieur.

\subsection{Compromission d'applications tierces}

\begin{flushright}\textit{Auteur: Rémy Léone}\end{flushright}

Les réseaux sociaux tirent leur source de revenus en utilisant majoritairement
les informations fournies par les utilisateurs pour afficher de la publicité
ciblée. Les réseaux sociaux peuvent aussi mettre à profit les informations de
leurs utilisateurs pour des applications tierces. Des éditeurs peuvent ainsi
concevoir des applications utilisant les API des réseaux sociaux afin d'extraire
des informations précises sur les utilisateurs, tel que leur date de naissance
ou bien leur liste de contacts.

Dans le cas de Facebook, des applications et des jeux sont proposés aux
utilisateurs. Bien que les applications soient forcées d'indiquer les
informations qu'elles vont extraire des profils utilisateurs qui les installent,
il est courant de voir des utilisateurs concentrés sur l'utilisation de
l'application plutôt que sur la sécurité de leurs données personnelles.  Ainsi
de nombreuses applications conçues sans véritables souci de la confidentialité
des données sont utilisées et peuvent être victimes de détournement
malveillants.

Les applications tierces ne sont pas soumises à un audit par le réseau social.
Ainsi, les vulnérabilités des applications tierces peuvent conduire à des fuites
d'informations concernant les utilisateurs qui les ont installées. En outre,
elles peuvent également héberger du code malveillant qui peut être téléchargé
sur la station de l'utilisateur et ainsi compromettre d'autres données
personnelles.

\subsection{Fuite de données}

\begin{flushright}\textit{Auteur: Patrick Raad}\end{flushright}

Un sondage révèle les risques qui sont associés aux réseaux sociaux
\cite{poll-dataleak}: 5,4 millions d’utilisateurs ont répondu à des e-mails de
phishing avec leurs informations personnelles. Parmi les utilisateurs adultes
des réseaux sociaux, 38\% ont publié leur date de naissance en entier, 45\% des
personnes ont publié les photos de leurs enfants, et 8\% ont publié le numéro de
la rue de leur adresse. Il est estimé que 5,1 millions ont été victimes
d'attaques ou d'abus en 2010, y compris des infections de logiciels
malveillants, les escroqueries et l’harcèlement.

Une autre étude faite en 2008 montre que moins de 1\% des fuites de données sont
dues à des attaques externes à l’entreprise.  \cite{dlp-whitepaper}  Ainsi ce
sont les utilisateurs internes qui, par malveillance ou par inadvertance,
dévoilent des informations confidentielles. Les exemples d'utilisation légitime
de réseaux sociaux sont nombreux. Le département des ressources humaines utilise
des réseaux sociaux professionnels pour chercher des candidats, et les équipes
de R\&D peuvent publier des guides de développement sur des wikis.  Cependant,
les réseaux sociaux introduisent plusieurs problèmes. Selon une étude faite par
Phenomenon Institute sur la sécurité des réseaux sociaux
\cite{social-media-risks}, l’utilisation des réseaux sociaux dans les
entreprises ont ajouté de nouvelles menaces de sécurité :

\begin{description}

\item[Virus \& logiciels malveillants :] Les infections se produisent quand les
employés téléchargent des fichiers et des données sur les ordinateurs de
l’entreprise;

\item[Fuite de données confidentielles :] La fuite des données a lieu quand les
employés envoient accidentellement ou intentionnellement des informations
sensibles à l’extérieur du réseau local. \cite{combating-data-leaks} Cette
solution peut faire la différence entre les informations personnelles,
confidentielles et celles de l’entreprise dans les réseaux sociaux, n’affecte
pas les communications légitimes (conformes avec les chartes du réseau social),
et utilise un système d’analyse de contenu en temps réel. Les vecteurs de fuites
d'informations sont multiples, messagerie, applications tierces, partage de
profil via des ``badges'' sur des sites externes\ldots

\end{description}

La DLP permet d'analyser l'ensemble des informations transitant au sein d'un
réseau afin de permettre à des renseignements stratégiques de ne pas quitter
l'organisation. Pour détecter le contenu, la DLP analyse l'ensemble des textes
voyageant sur les liens surveillés. Dès qu'un contenu a une signature similaire
à une signature stockée en mémoire, le système bloque la communication et envoie
un message à l'administrateur système. La messagerie étant un vecteur faisant
transiter énormement d'informations, elle est souvent la première surveillée.
Le choix d’une solution de Data Leak Prevention est basé sur plusieurs critères :

\begin{itemize}

\item Il est nécessaire pour les DLP d'avoir une base de données de signatures à
jour afin de permettre une détection efficace.

\item Pour être efficace, les DLP doivent être au courant de l'ensemble des
services de réseaux sociaux, ainsi que des serveurs mandataires qui peuvent
permettre de contourner les règles de sécurité.

\item Avec l’évolution des réseaux sans fil, et des terminaux tel que les
ordinateurs portables et les smartphones, le périmètre de sécurité propre à
l’entreprise est facilement contournable. Par exemple, un employé peut vouloir
accéder à Internet depuis l’extérieur du périmètre de son entreprise, continuer
son travail depuis son domicile, ou travailler depuis l’étranger via la station
appartenant à l'entreprise et contenant des informations confidentielles.  Pour
cela, un agent DLP doit être installé sur le terminal afin de pouvoir
anticiper les éventuelles fuites vers l'extérieur. 

\end{itemize}

Il est également à noter que les réseaux sociaux sont filtrés, afin de lutter
contre la baisse de productivité que les employés sont susceptibles de causer
lorsqu'ils les consultent trop souvent. Ainsi, les réseaux sociaux sont
souvent intégrés dans les politiques de sécurité des entreprises en étant
purement et simplement bloqués. À l'inverse, certaines sociétés essayent
d'utiliser un réseau social interne comme unique moyen de communication, et
évitent ainsi les mails.


% vi:et:ts=2:sw=2:tw=80:
