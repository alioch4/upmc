\section{Tableau comparatif}

\vspace{-1.30cm}

\begin{sidewaystable}[H]
	\begin{tabular}{|p{2.7cm}|p{3.2cm}|p{3.6cm}|p{3.4cm}|p{2.9cm}|p{3.0cm}|}
\hline
\textbf{Critères}
& \bf Facebook
& \bf Google+
& \bf Linkedin
& \bf Vis-à-vis
& \bf Safebook
\\ \hline

\bf Contrôle d'accès
& \multicolumn{2}{p{6.9cm}|}{\centering Liste de contrôle d'accès sur chaque information appartenant à un utilisateur}
& Mode privé, restreint ou public pour chaque élément du profil
& Confiance dans l'hébergeur
& Total
\\ \hline

\bf Anonymat des interactions
& Impossible (Cas des Like par exemple)
& Impossible (Cas des +1 par exemple)
& Impossible (Cas des groupes publics)
& Possible
& Possible
\\ \hline

\bf Applications tierces
& Aucun contrôle sur les applications tierces
& Quelques jeux, mais il n'est pas pour le moment possible de créer arbitrairement une application sans le contrôle de Google
& Les applications tierces sont peu nombreuses et triées sur le volet par Linkedin
& Inconnu
& Inconnu
\\ \hline

\bf Fuites de données
& \multicolumn{3}{c|}{Aucune fuite de données majeure connue à ce jour}
& \multicolumn{2}{c|}{Dépend de la plateforme de l'utilisateur}
\\ \hline

\bf Synchronisation des données
& \multicolumn{3}{c|}{Immédiate}
& Relativement rapide
& Cas de latence
\\ \hline

\bf Disponibilité
&\multicolumn{3}{c|}{Système fonctionnant via des datacenters très performants.}
& Disponibilité élevée (Amazon EC2)
& Dépend de la station sur laquelle on est installé
\\ \hline

\bf Anonymat
&\multicolumn{3}{c|}{Autorité centrale, anonymat impossible pour les utilisateurs.}
&Confiance dans l'hébergeur
& Possible
\\ \hline

\end{tabular}
\caption{Comparatif des cinq réseaux sociaux étudiés}
\label{tab:comparatif}
\end{sidewaystable}
