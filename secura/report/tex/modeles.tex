\section{Modèles}

\begin{flushright}\textit{Auteur: Patrick Raad}\end{flushright}

Afin d'étudier la sécurité et la protection de l'intimité au sein des réseaux
sociaux, il est important de disposer de modèles précis des notions
d'\textit{intégrité}, de \textit{disponibilité} et d'\textit{intimité}.

\begin{description}

\item[Intégrité :] On appelle \textit{intégrité des données} l'impossibilité
pour un utilisateur de pouvoir modifier les données d'un autre
utilisateur. \cite{safebook}  Ceci englobe la protection de l'identité et des
informations concernant lui et ses contacts.  En outre, pour remplir cet
objectif, il faut également que les communications entre les différents
utilisateurs soient chiffrés de bout en bout.

\item[Disponibilité :] La \textit{disponibilité} vise à assurer le
fonctionnement des services du réseau social malgré les attaques et les fautes.
L'indisponibilité peut nuire à la réputation, surtout dans le cas des réseaux
sociaux à orientation professionnelle.

La décentralisation peut améliorer la disponibilité dans de nombreux cas de
figure. Les datacenters des réseaux sociaux centralisés utilisent plusieurs
nœuds dans leur réseau interne, afin d'assurer une disponibilité constante, même
en cas de pannes. Ils agissent comme des nœuds d'un réseau décentralisé qui
synchronisent entre eux le contenu, afin de garantir la cohérence des données
pour leurs utilisateurs.

De tels réseaux décentralisés sont aussi utilisés dans le cas des réseaux
sociaux décentralisés de taille plus modeste. Il est ainsi possible d'avoir
une copie de données des utilisateurs quand leur nœud maître est absent.

L'architecture décentralisée permet d'augmenter la résilience totale d'un
service. Son utilité est démontrée quand un nœud indisponible peut être
remplacé par un nœud voisin qui aura une sauvegarde relativement récente de ses
données. Pour un réseau social décentralisé, une bonne disponibilité est
cruciale, car les réseaux sociaux centralisés ont une force considérable dans
ce domaine. \cite{facebook-uptime}

Il est à noter que la disponibilité est une notion assez subtile qui est
fortement liée aux performances du réseau social. Ainsi, dans les architectures
Web modernes, \cite{youtube} il est assez courant de voir des mécanismes de file
d'attente qui valideront l'envoi d'une vidéo quand des ressources seront
disponibles pour l'encoder par exemple.  Ainsi, même si la vidéo est envoyée,
elle ne sera disponible qu'après un certain laps de temps.


\item[Intimité :] On appelle \textit{intimité} la capacité d’un utilisateur à pouvoir
contrôler, protéger et divulguer les informations qui le concerne qu'aux
personnes qu'il choisit explicitement  Plusieurs attaques la menacent :

\begin{description}
\item[Vol d'identité :] Un membre malveillant ou un service obtient l’accès au
compte d’un membre et contrôle ses informations et sa liste d’amis;

\item[Usurpation d'identité :] La création d’un compte en utilisant l’identité
d’une personne pour obtenir la confiance de ses connaissances.

\item[Profils illégitimes :] Un compte est créé par un utilisateur n'étant pas
légitime pour avoir un profil ciblé. Cette attaque prend tout son sens dans un
contexte de campagne politique, ou bien dans un milieu professionnel.

\end{description}
\end{description}

Généralement quand on parle de l’intimité dans le contexte d’une
communication entre deux individus, on souhaite garantir
quatre propriétés: \cite{handbook}

\begin{description}

\item[Contrôle d'accès :] L'utilisateur a le contrôle total aussi bien en
lecture (consultation) qu'en écriture (ajout de commentaires, modification d'un
contenu, \ldots) sur un contenu qu'il publie. En particulier, il peut modifier
ces permissions pour des ensembles précis d'autres utilisateurs du réseau
social, ainsi que pour des utilisateurs non identifiés;

\item[Anonymat :] Les utilisateurs doivent accéder aux ressources
sans divulguer leur identité;

\item[Non-observabilité :] Aucune partie tierce ne doit pouvoir recueillir
d'informations à propos des communications et des échanges de contenu entre des
utilisateurs;

\item[Non-recoupement :] Dans le cas de l’obtention de deux messages par un
tiers, il est impossible de déterminer si les deux messages sont envoyés par le
même expéditeur ou reçu par le même destinataire;

\item[Non-traçabilité :] Aucune partie tierce ne doit pouvoir garder de traces des
actions des utilisateurs dans un système, ce qui exige l'anonymat et le
non-recoupement d'informations.

\end{description}
