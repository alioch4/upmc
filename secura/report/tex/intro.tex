\section{Introduction}

\begin{flushright}\textit{Auteur: Patrick Raad}\end{flushright}

Les utilisateurs des réseaux sociaux en ligne sont de plus en plus concernés
non seulement par le partage de leurs informations, mais aussi par la
protection de leurs données. La plupart des sites de réseaux sociaux sont de
grandes entreprises qui sont soumises aux lois des différents pays. Chaque pays
peut imposer ses règles à ces entreprises. L'entrée de Facebook en Chine a
nécessité la signature d'un contrat avec le géant du Web chinois Baidu,
prévoyant la création d'un nouveau réseau social respectant les règles de la
censure. \cite{facebook-baidu}

Les réseaux sociaux suivent des mécanismes bien précis: tout d'abord, les
utilisateurs doivent accepter des conditions d'utilisation qui déterminent les
interactions autorisées d'un utilisateur avec le réseau social. Puis
l'utilisateur doit accepter une charte décrivant la protection de ces
informations et comment ses données personnelles seront utilisées afin de lui
proposer du contenu adapté à son profil. Une fois que les chartes d'utilisation
sont acceptées, l'utilisateur a la possibilité d'ajouter du contenu et des
informations personnelles et filtrer les accès à ces données selon une
politique qu'il aura déterminé.

De nombreux utilisateurs ne lisent pas les conditions d'utilisation des sites
et acceptent la charte sans se poser de questions.
\cite{facebook-privacy-issue} Pour un grand nombre d'entre eux, ils ne savent
pas comment leurs informations et leurs données personnelles sont utilisées.
\cite{facebook-user-privacy} En outre, les réseaux sociaux se réservent le
droit de modifier les conditions d'utilisation quand bon leur semble, et ainsi
changer l'usage qui est fait des données personnelles des utilisateurs. Ce qui
amène les décideurs politiques à intervenir sur ces sujets.
\cite{europe-facebook}  Ce problème est d'autant plus préoccupant que chaque
réseau social semble avoir ses propres règles en la matière. Pour
l'inscription, chaque réseau social exige un ensemble de données différent,
allant du plus trivial, comme le nom ou le prénom, jusqu'au plus intime, comme
l'orientation sexuelle pour certains.  \cite{privacy-jungle}

Par exemple, récemment, Google a modifié sa charte de vie privée afin de leur
permettre d'utiliser les données recueillies par ses services avec une plus
grande flexibilité, en croisant les informations recueillies par tous les
services de Google, afin d'établir un profil utilisateur. \cite{spyw} Cela veut
dire que YouTube, Gmail, Blogger, Google+ et Web History vont pouvoir
communiquer entre eux, et catégoriser l'utilisateur selon ses préférences. Cela
peut donner à l'utilisateur un certain confort dans ses recherches sur
Internet. Cependant, en poussant le recoupement de données à l'extrême, on
pourrait alors apprendre tout sur une personne (son nom, ses points de vue
politiques et religieux, etc.). En prenant une photo d'une personne, grâce aux
techniques de ``tagging'' et de reconnaissance faciale, il serait possible d'avoir
toutes les informations le concernant. Dans ce cas, la question de l'intimité
sur Internet est remise en cause.

Les chercheurs essaient de trouver une solution à cette accumulation de
données, en se tournant vers les réseaux sociaux distribués pair à pair et en
améliorant les politiques de sécurité des réseaux centralisés existants.

Dans la section 2, nous exposerons les modèles et comment ils sont appliqués
dans les cinq réseaux sociaux que nous avons retenus pour notre étude; dans la
section 3, nous discuterons de quelques attaques et types d'attaque précis
appliqués aux réseaux sociaux; enfin, nous traitons en section 4 les scénarios
possibles d'évolution des réseaux sociaux et leurs politiques de sécurité.
