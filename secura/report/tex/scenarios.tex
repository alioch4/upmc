\section{Scénarios possibles}

\begin{flushright}\textit{Auteur: Rémy Léone}\end{flushright}

De nombreuses pistes d'évolution existent pour les réseaux sociaux et leurs
politiques de sécurité. Il paraît clair que les réseaux centralisés existent et
seront de plus en plus puissants dans les années futures.

Afin d'équilibrer la tendance des réseaux sociaux à demander toujours plus
d'informations concernant les utilisateurs, il serait souhaitable de mettre en
place des autorités à but non lucratifs. TRUSTe a démontré la pertinence d'un
programme en charge de surveiller et d'accorder des licences qui certifient
qu'un site web protège et gère convenablement les informations de leurs
utilisateurs.

Une des critiques à l'encontre des réseaux sociaux décentralisés est l'absence
d'autorité de contrôle des identités. Elle est très utile dans les cas
d'attaques par faux profils qui envoient du contenu parfois douteux ou
diffamatoire sur le réseau. Afin de lutter contre ce phénomène, il serait
possible de mettre en place des autorités en utilisant des adresses DNS. En
effet, dans de nombreux pays la réservation d'un nom de domaine doit
impérativement passer par la publication d'informations légales faisant foi en
cas de litiges judiciaires. En utilisant les informations données à l'autorité
DNS au moment de l'inscription il serait possible d'instaurer un contrôle
supplémentaire sur les profils inscrits.

Cependant, malgré leur prolifération, les solutions d'interopérabilité sont
encore peu présentes. Il est pour le moment relativement difficile d'avoir des
applications pouvant partager facilement sur de multiples réseaux sociaux une
seule et même information avec différents modèles de sécurité. Cette
fonctionnalité pourrait aider à diffuser des informations sous une même
interface avec des politiques de sécurité précises sur des réseaux sociaux
différents aussi bien centraux que décentralisés.
