\section{Analysis of scientist conversation distribution}

\subsection{Experimental method}

We used several script to extract information from the data provided before this study.
First we read the main data using a python script, then we process it by creating some 
new variables (total time of conversation, mean time of conversation). Then we normalize
the data using R programming language. Finally we use gnuplot to create graphical representation
of our data.

\subsection{Results analysis}

We obtain a very heterogeneous set of results. Several behavior could be observed in our result :
\begin{description}
\item Average number of connections with other scientists
\item Lower than average connections
\item Above average connections
\end{description}

Most of the scientists attending to this conference seems to have a very average behavior
(having around 125 conversations). Nevertheless, some of them are having more conversations
and during a longer period. It's clear by looking at the deviation from the mean value of the 
number of conversation and the mean time of conversation. We can also noticed some very low
results from some scientists. There is some outliers, two of them are a complete lack of conversations
giving two missing point, and one of them is a high values chatting person.

\begin{figure}[h]
\begin{center}
\includegraphics{img/part1}
\end{center}
\caption{Normalized results of conversation}
\end{figure}

We can also notice that some scientists are not talking at all, because we don't know 
a lot about how this data was created we can suppose it could be some errors.

\subsection{Conclusion and open questions}

Other informations could be required to analyse more deeply this conference and answer more questions. 
For instance, we don't 
have any information about the scientists. It could be very interesting to compare this conversations
with the administrative position of each of them (graduate student, senior, laboratory leader).

A study of the chat distribution could be interesting. Suppositions can be made about the correlation
between the administrative responsability and the chat-appeal a person could have in a science conference.

Some results could be obtained by looking at the time distribution, maybe there is some period where
no communications appends and some others that are very dense. Linking this results

\lstinputlisting[language=Gnuplot]{scripts/script.gnuplot}

