\documentclass{sig-alternate-10pt}

\begin{document}

\title{Homework - Making the point}

\numberofauthors{1} %  in this sample file, there are a *total*

\author{
  \alignauthor
    Rémy Léone\\
    \affaddr{UPMC~-- M2 Recherche~-- METHO}\\
    \email{remy.leone@gmail.com}
}

\maketitle

\section{Analysis of scientist conversation distribution}

\subsection{Experimental method}

We used several script to extract information from the data provided before this study.
First we read the main data using a python script, then we process it by creating some 
new variables (total time of conversation, mean time of conversation). Then we normalize
the data using R programming language. Finally we use gnuplot to create graphical representation
of our data.

\subsection{Results analysis}

We obtain a very heterogeneous set of results. Several behavior could be observed in our result :
\begin{description}
\item Average conversation and connection with other scientists
\item Lower than average communication
\item Above average communication
\end{description}

\begin{figure}[h]
\begin{center}
\includegraphics{img/part1}
\end{center}
\caption{Normalized results of conversation}
\end{figure}

We can also notice that some scientists are not talking at all, because we don't know 
a lot about how this data was created we can suppose it could be some errors.

\subsection{Others questions}

Other informations could be required to analyse more deeply this conference. For instance, we don't 
have any information about the scientists. It could be very interesting to compare this conversations
with the administrative position of each of them (graduate student, senior, laboratory leader).

A study of the distribution of chat could be interesting. Supposition can be made about the correlation
between the administrative responsability and the chat-appeal a person could have in a science conference.


\section{Availability in large scale distributed networks}

SETI@home project was launched in May 1999 and have used to study the behavior of 
large scale distributed network (\cite{ja_ko_mascots09}). The main problem of SETI@home
(and all benevolant projects alike) is the availability of node that could evolve very quickly.
As a result the average availability could suffer from a massive disconnection of nodes.

We could compare the average result of a network with similarity with SETI@home and a classical
network (A supercomputer or a cluster of university's computers).

\subsection{Experimental method}

We used the massive amount of data that were used by previous work on distributed computing (\cite{ja_ko_mascots09})
to calculate an average computing power. 
\bibliographystyle{abbrv}
\bibliography{main}


\end{document}
