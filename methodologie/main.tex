%!TEX TS-program = xelatex
%!TEX encoding = UTF-8 Unicode
\documentclass[11pt, a4paper]{article}

\hypersetup{
    backref=true, %permet d'ajouter des liens dans...
        hyperindex=true, %ajoute des liens dans les index.
        colorlinks=true, %colorise les liens
        breaklinks=true, %permet le retour à la ligne dans les liens trop longs
        urlcolor= blue, %couleur des hyperliens
        linkcolor= blue, %couleur des liens internes
        bookmarks=true, %créé des signets pour Acrobat
        bookmarksopen=true, %si les signets Acrobat sont créés,
        %les afficher complètement.
            pdftitle={XDexplorer - Slides}, %informations apparaissant dans
            pdfauthor={Remy Leone}, %dans les informations du document
            pdfsubject={XDexplorer, P2P, sciencepapers, education} %sous Acrobat.
}

%\AtBeginSection[]{
%    \begin{frame}{}
%    \small \tableofcontents[currentsection, hideothersubsections]
%        \end{frame} 
%}

\include{commands}

\begin{document}

\author{Leone Rémy}
\title{A Survey of Green Networking Research}
\subtitle{Aruna Prem Bianzino, Claude Chaudet, Dario Rossi, Jean-Louis Rougier}
\date{\today}
\institute{GT2 - UPMC}

\section{How did I select this paper ?}

I used several parameters to select this article. My main requirement was
my interest in the subject. It seems clear that the best results come
from passionate people who have an interest in the subject.

I had other criteria based on the possible scenario of implementation of
the project. P2P trackers are widely available today. They spread  a lot of files, 
in a very efficient way. It could be possible to adapt them to
spread scientific articles. Various people could be interested in a project similar
to this one. For instance universities wanting to spread theses and articles from
fellow students around the world and collect various interesting and well ranked papers
coming from other universities.

The entire project will be released under GPLv3 licence to be sure it 
remains a free software. Ultimatly, the goal is to provide an efficient way to exchange 
educational contents easily and for free between academia, students and professors.

\section{Description of my project}

Creating a completly distributed science paper repository would be an interesting project 
for academic and industrial research \cite{overcite}. The maintenance could be easier and faster
compared to the cost of maintining several sites up to date. In case of
failure other copies would be available, the downloading of contents would be faster than from
a single server and it could lead to better mapping of scientific articles. If a large and significant
amount of papers are all gathered in a free and open platform, statistical mapping of research would
be efficient and could lead to a new collaboration between the industry and academia.

Science papers aren't the only content concerned by this system. The global education
system could be enhanced by using a free and efficient educationnal content used
in a lot of countries. This could lead to a better and cheaper globalization of knowledge.
\bibliographystyle{alpha}
\bibliography{main}


\end{document}
