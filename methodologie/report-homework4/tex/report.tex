\section{Introduction}

The software diversity in order to manage reference is already
great but no software is completly satisfactory. Scientific and
educational contents are spread in the Internet and there is currently
no way to gather them into a single and simple distributed network of repository.

This paper aims to draw what could be a good solution in order to
spread educational contents and scientific articles in a distributed
way.


\section{State of art \& Related works}

\begin{itemize}

\item \textit{P2P is a mature and production ready technology.}

\item \textit{PDF indexing and searching techniques are mature and quickly growing.}

\item \textit{Need to have a coherent and accurate interface for science articles.}

\end{itemize}

\begin{itemize}

\item \textit{Google Scholar is good but it doesn't provide an API and works as a black box.}

\item \textit{Not everyone uses a MacOSX but everyone should have a good user interface experience.}

\item \textit{Web-based solutions are better for interoperability.}

\item \textit{Education should be easily accessed all over the world.}

\item \textit{Some initiatives already exists but not a single open network is massively used.}

\end{itemize}

\section{Problem statment}

\begin{itemize}
\item \textit{Distributed way to spread education contents like
scientific articles, education manual, \ldots}
\item \textit{P2P network with no central authority. Everyone should be
able to quickly finds accurate science articles for free}
\end{itemize}

\section{Usage}

\textit{Some initiatives already exists}

\begin{itemize}
\item OverCite
\item Open Archive initiative
\item arXiv.org
\end{itemize}

\section{Contributions}

\subsection{Indexation}

\subsubsection{Bib\TeX{} analysis and reference}

\textit{Indexing, indexing, indexing\ldots}

\subsubsection{No double utility}

\textit{Thanks to a hash mechanism, double files
will be avoided}

\subsection{Repository}

\subsubsection{A web interface to browse contents}

\textit{A stable web interface is required in order to
quickly deploy good web interface in order to make the system
appealing.}

\subsubsection{Security issues}

\textit{In this part we will see how we can make connections
between peers safer by an authentification process. With this
system only validated papers can be broadcasted on the network.}

\subsection{Application}

\subsubsection{Different ranking methods}

\textit{Reference and ``must read papers'' along with fresh new papers}

\subsubsection{Simplified accounts}

\textit{Only hashes and file pointers are needed in order to make an
account to memorize the preferences. We can have a account management
website with very few ressources.}

\subsubsection{RSS utilities}

\textit{Scientists should have all the information that matters directly into
one feed.}

\subsubsection{Graphical user interface to manage the current personnal repository}

\textit{The complete personnal repository is manageable by using a command line
tool or by using a graphical user interface}

\subsubsection{Meta information about files}

\textit{Index files wherever or whatever they are. Indexing sections, subsections
and keywords. Indexing everything that matters.}

\subsubsection{Files distance}

\textit{In order to suggest articles that might be interesting we can use
some keywords parsing in order to see what are the contents that may be more
relevant to a user}

\section{Results}

\textit{Insert here some statistics results about what 
it would require to manage an arXiv-scale website with P2P technologie}

\section{Conclusion}

\textit{Additionnal time and administrative power is required in order to
make this project become reality}

