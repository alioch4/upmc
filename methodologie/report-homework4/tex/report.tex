\section{Introduction}

The software diversity in order to manage reference is already 
great but no software give a complete satisfaction. Scientific and
educationnal content are spread in the Internet and there is currently
no way to gather them into a single and simple distributed network of repository.

This paper aims to draw what should be a good solution in order to 
spread educationnal content and scientific article in a completly distributed 
way.


\section{Problem}

\textit{GOOGLE SCHOLAR DON'T HAVE API}

\section{Usage}

\subsection{OverCite}

\section{Indexation}


\subsection{Bib\TeX{} analysis and reference}

\textit{Indexing, indexing, indexing\ldots}

\subsection{No double utility}

\textit{Avoid to have several time the same article will
be easy with an hash mechanism}

\section{Repository}

\subsection{A web interface to browse content}

\textit{A stable web interface is required in order to
deploy quickly good web interface in order to make the system
appealing.}

\subsection{Security issues}

\textit{In this part we will see how we can make connections
between peers safer by an authentification process. With this
system only validated papers can be broadcasted on the network.}

\section{Application}

\subsection{Different ranking methods}

\textit{Reference and ``must read papers'' against fresh new papers}

\subsection{Accounts made light}

\textit{Only hashes and file pointers are needed in order to make an
account to memorize the preferences. We can have a account management
website with very few ressources.}

\subsection{RSS utilities}

\textit{Scientists should have all the information that matters directly into
one feed.}

\subsection{Graphical interface to manage the current personnal repository}

\textit{The complete personnal repository is manageable by using a command line
tool or by using a graphical interface}

\subsection{Meta information about files}

\textit{Index files wherever et whatever they are. Indexing sections, subsections,
keywords. Indexing everything that matters.}

\subsection{Files distance}

\textit{In order to suggest articles that might be interesting we can use
some keywords parsing in order to see what are the contents that may be more
relevant to an user}
