\section{Seminal article on search over P2P networks}

My project deals with the realization of a P2P system for
scientific papers and more generally education contents.
Searching needs to be fast and accurate in such system.
The seminal paper of the searching over P2P system is Yang's article \cite{Yang02improvingsearch}

\subsection{Confidence}

I am not an expert. My evaluation is that of an informed
outsider. I may miss some flaws in the work.

\subsection{Summary}

\paragraph{Type of paper}

This paper aims to improve the current performance of search in P2P systems.
In order to do this, three techniques are presented : Iterative deepening, local
indices and the directed breadth-first traversal. The article brings ameliorations
but not any new concepts to P2P systems.

\paragraph{Context for this paper}

This paper was written in 2002. At this time Gnutella, Napster and Morpheus were the
very well known P2P systems in production. Searching techniques on these systems
were inefficient. In order to enhance this, the authors proposed to use different
searching techniques in order to speed up the search and make it more accurate.

P2P systems were not as known and used as they are today. This paper is really
one of the first one to raise up some questions and research about search in P2P
systems.

\paragraph{Correctness}

As the experimentations shows it, the search improvements are working.

Nevertheless, we can also notice a tradeoff between the Time to Satisfy
and the other criteria. Maybe some recent work would be interesting
in order to make this tradeoff better.

\paragraph{Contributions}

The contributions of this paper are good improvements of the global quality
of search in P2P system. Even if the details about implementations in
currently used system, such as Bit Torrent or DC++, are not known, the ideas
are simple to implement within the systems. It seems clear that the ideas
in this paper contribute to the whole P2P community.

\paragraph{Understandability}

This article is very clear and the ideas used to present the
amelioration of the searching techniques are well exposed.

\subsection{Strengths}

% What are the paper's main strengths?

The main strength of this paper is to boost up performance
of searching in P2P. When this paper was released, search
was inefficient and very costly. This paper enhances the speed
and reduced the cost in bandwidth. Another strengh of this paper
is also to propose simple ways to make search in P2P systems better
without taking the particularities of any already existing systems.
Even if the Gnutella network was the only one used inside the experimental
process, other P2P systems can use these techniques that are designed for
P2P systems. As a general article the scope of this solution is quite
large.

\subsection{Weaknesses}

% What are the paper's main weaknesses?

This paper has gnutella as the main experimental data. Today, Bit Torrent is the main
P2P system in use. It could be interesting to see how modern tracker technologies could
enhance the strategy presented in this article.

Also, we don't know how the use of a DHT system could also help this searching techniques.

\subsection{Evaluation}

This is among the best I've reviewed. I'd really want to
see this paper in ConEXT and will champion it if I am in
the PC.

\subsection{Detailed Comments}

% Contribution

The contribution of this article is three techniques, potentially usable to improve the
quality and speed of search in a P2P system.

% Technical merit

The field of search in ``loose'' P2P systems was small when this article was released.
So this article actually succeeded in a widely open field . We can also mention the fact
that with iterative deepening, the authors bring concepts that are already working in artificial
intelligence fields.

% Writing

The writing style of this article is very clear, the annexes help to understand
how the calculation are made. It can be very helpful in order to recreate the experimental
setup to see if the values are still accurate nowadays.

The authors also pay attention to numerical examples that help to
 emphasize the scale of enhancing.

% => How to improuve the paper

As all seminal papers, it's really important to see how the results announced are still accurate
to describe current systems. The search techniques presented in this article are easy to integrate
in P2P clients and should give enhancing searching results quickly.

\subsection*{End Review}

\pagebreak
