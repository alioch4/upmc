
\section{Seminal article on search over P2P networks}

My project deals with the realization of a P2P system for
scientific papers and more generally education contents.
In order to provide an efficient and accurate system of
education delivering, search among this project must be
accurate.
The seminal paper of the searching on P2P system is
the Yang's article \cite{Yang02improvingsearch}

\subsection{Confidence}

I am not an expert. My evaluation is that of an informed
outsider. I may miss some flaws in the work.

\subsection{Summary}

\paragraph{Type of paper}

This paper aims to improve the current performance of search in P2P systems.
In order to do this, 3 techniques are presented : Iterative deepening, local
indices and the directed breadth-first traversal. The article bring ameliorations
but not really a new concept of P2P systems.

\paragraph{Context for this paper}

This paper was written in 2002 at this time Gnutella, Napster and Morpheus was the
very well kwown P2P systems in production. Searching techniques on this systems
was inefficient. In order to enhance this, the authors proposed to use different
searching techniques in order to speed up the search and make it more accurate.

\paragraph{Correctness}

As the experimentations shows it, the search improvement are working.
But we can also see that there is a tradeoff between the Time to Satisfy
and the other criteria. The tradeoff shows a clear improvement the cost
in bandwidth and a global decreasing of the aggregate cost of processing.


\paragraph{Contributions}

The contributions of this paper are good improuvment of the global quality
of search among P2P system. Even if the details about implementations in
currently used system (like Bit Torrent or DC++) are not kwown, the ideas
are simple to implement inside the systems.

\paragraph{Understandability}

This article is very clear and the ideas in order to present the
amelioration of the searching techniques are well exposed.

\subsection{Strengths}

% What are the paper's main strengths?

The main strength of this paper is to boost up performance
of searching in P2P. When this paper was released, search
was inefficient and very costly. This paper enhance the speed
and reduced the cost in bandwidth. An other strengh of this paper
is also to propose simple way to make search in P2P systems better
without taking the particularities of any already existing system.
Even if the Gnutella network was the only one used inside the experimental
process, other P2P systems can use this techniques that are designed for
P2P systems.

\subsection{Weaknesses}

% What are the paper's main weaknesses?

This paper have gnutella as main experimental data. Today, Bit Torrent is the main
P2P system use. It could be interesting to see how modern tracker technologies could improuve
or enhance the strategy presented in this article.

Also, we don't know how the use of a 
\subsection{Evaluation}

This is among the best I've reviewed. I'd really want to
see this paper in ConEXT and will champion it if I am in
the PC.

\subsection{Detailed Comments}

% Contribution

The contribution of this article is three techniques usable to improve the
quality and speed of search among a P2P system.

% Technical merit

The field of search among ``loose'' P2P systems was small when this article was released.
So this article actually succeed in a field widely open. We can also mention the fact
that with iterative deepening, the authors bring concepts that are already working in artificial
intelligence fields.

% Writing

The writing style of this article is also very clear, the annexes help to understand
how the calculation are made. It can be very helpful in order to recreate the experimental
setup to see if the values are still accurate nowadays.

A big place is also allowed to numerical

% => How to improuve the paper

As all seminal papers, it's really important to see how the results announced are still precise
to describe current systems. The search techniques presented in this article are easy to integrate
in P2P client and should give enhancing searching results quickly.

\subsection*{End Review}

\pagebreak
