\section{Survey on distributed delivery system}

% What is the problem
% What are the contributions
% What are the conclusions
% What is the support for the conclusions

This paper will attempt to pictures what are the most influencial papers of Overcite article
\cite{overcite}. I look among all the papers listed in the references to decide whether 
or not a paper was related to it. The reference list was 49 articles long. I used the 
amount of external citations to decide if an article was well known enough. So
according to google scholar the article cited in this present paper are all above several 
hundreds time cited in other papers. This way of grading the articles is appropriate 
for finding a seminal paper (which is the point of the second part of this paper) but 
it's not a very efficient paper to evaluate the new and not yet well referenced papers.
Hence, it's possible than more recent and accurate papers influenced Overcite but are 
currently not known enough.

\subsection{Bit Torrent, a major P2P system}

One of the most influencial paper about P2P technologies is the one written by Bram 
Cohen creating the Bit Torrent protocol. This paper gives all the basis of the Bit Torrent protocol.
The tit-fot-tat, the pareto efficient goal, the tracker concept. All of this concepts that seems natural
right now in the P2P world were introduced in this paper \cite{Cohen03incentivesbuild}.
We can see that there is no mention of the DHT in this paper, 

\subsection{Searching on P2P networks}

To build a really efficient P2P scientific articles network we need to also
ship a very efficient and scalable way of search information among this network.

Several way have been proposed to do it \cite{Yang02improvingsearch}. These methods give
very good results if we take the searching technique BFS as a baseline. But the status of implementation
of this techniques into current P2P systems is yet unknown. It could be interesting to see how much 
the idea of this article applied into a real production system.

Different way of providing information about the peers have been released one of the most known one is
the DHT (Distributed Hash Table). An implementation have been provided by Sean Rhea \cite{Rhea:2005:OPD:1090191.1080102}
but this implementation have been deprecated mostly because more mature services (like Amazon Dynamo and now Cassandra)
have been used massively by well known web actors. DHT implementation seems to be mature enough to support such large
services like Facebook or Amazon. So this paper contributed to the diffusion of the DHT concepts to a wider public.
Powered during his work by planet lab to test it, Sean Rhea added 
his work to build a good implementation \href{bamboo-dht.org}{Bamboo}.

Other techniques are also used to improve the quality of search in P2P networks, we can quote
\cite{Reynolds_efficientpeer-to-peer} in order to see that bloom filters, cache and incremental
results also enhance the quality and speed of the results. A very interesting study would be to
see how this techniques are integrated inside recent P2P technologies among cluster like Cassandra
or other system. There is not a single and united developement of P2P techniques so the advance of 
each project is very heterogeneous. Provide unique and very stable librairies to all this project would
be a sign coherent development among the P2P developers community.

\subsection{Indexing papers}

Browsing papers is not enough, making connections between them is really important to make the really 
important papers come out of the whole set. Accoring to this, a current and quite relevant way to
indexing content is the google Page Rank system. But citations is not only relevant, for instance 
well known articles are really cited a lot but sometimes they don't have any clear and deep link 
with the subject treated. Quoting very known articles allows to considered the paper as more mature
and more aware of all the current work. The Overcite project is really \cite{Lawrence99digitallibraries}
active and well referenced even google scholar use it as a first choice provider in most case 
for referenced scientific papers.

\pagebreak
