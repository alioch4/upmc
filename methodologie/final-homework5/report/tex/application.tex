\subsection{Application}

\subsubsection{Different ranking methods}

One of the main problem for a paper to be read is the reference. It's
quite hard for a new paper to be as well-referenced than an old one.

But new scientists need to read all the classical and seminal 
articles before start to read the most recent one.

In order to satisfy this two requirements, two ranking system will
be installed. One will be based on the amount of reference that an
article can get. And the other one with a mix of publication date,
reference in other article, and h-index of scientists.

The main goal here is to provide a quick and efficient way for
experienced scientists to read easily the new article of their fields
and for new one to get quickly the ''must read`` of their field.

\subsubsection{Simplified accounts}

\textit{Only hashes and file pointers are needed in order to make an
account to memorize the preferences. We can have a account management
website with very few ressources.}

\subsubsection{RSS utilities}

Stay informed of the current situation of research is a daily
activity for scientists. Conference give regulary good articles but
not all fields are exposed in the released papers. A paper coming from
a small university can be completly ignored if it's not referenced by
other people.

In order to stay in touch with the new article a RSS 
(Really simple syndication) will be provided. It will allow scientists
to follow a specific field of research by using hashtags and a system
of keywords.

\subsubsection{Graphical user interface to manage the current personnal repository}

System administrors like to manipulate application with script to be update, installed
systematically. In order to do that, we must provide a command line interface in order
to update and configure easily the application by scripting.

Most of scientists are not computer science scientist and don't have a very deep (or interest)
for computing. In order to provide them a good user experience we must also provide a good 
graphical interface. Because users are a critical ressource for our P2P network we 
must lure in a lot of user to have an efficient P2P system. Good example
of tools can be found on the MacOSX platform but all of them
are not standard. Also the web seems to be a very good alternative
to all "platform" solution. By using a web-browser in order
to provide graphical interface to system. All the platform-dependant
difficulties are skipped to provide quickly a good interface whatever
the platform is.

\subsubsection{Meta information about files}

The main interest of our network is to provide an efficient way to parse and
index all the content of a file in order to only share meta-information about the
file but not the file itself. In fact PDF file must be heavy (especially if
they content figures, illustrations,\ldots) in order to avoid useless network usage,
only meta-information will be shared with other peers at the beginning. When a file
will be requested by other peers, the real PDF file will then be sent on the network.

If the repository are mirroring all the time other repository both (meta information and
PDF file) will be sent to the requesting peer.

\subsubsection{Files distance}

\textit{In order to suggest articles that might be interesting we can use
some keywords parsing in order to see what are the contents that may be more
relevant to a user.}
