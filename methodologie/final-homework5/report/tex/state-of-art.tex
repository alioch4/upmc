\section{State of art \& Related works}

There is currently no problems for a scientist to find a precise article.
A simple search on the web most of the time is enough to quickly get to a precise
article. 

There is also some initiative that aims to index science papers in order
to provide meaningful results to queries from scientists for instance Google Scholar,
arXiv.org, Overcite, \ldots 

But there is not a well established way for exchanging all this information
among a massively used decentralized P2P system. However, it's also impossible for some of this service to
request them without using their own system (For instance Google Scholar which works
as a black box). All this limitation
points out the interest of a really decentralized, free and open P2P system to deliver science 
article.

P2P is a mature technologie currently used to exchange various information on a very large
scale. It could be relevant to aim for a really useful and free tool in order to spread knowledge and
innovation on a global scale. The main interest of P2P system is to provide fast downloads of contents.
If a certain amount of universities, schools and simple uers are interested into getting a lot of science contents
for free on their stations then an efficient P2P network can be built.

Accurate search is also a problem in this kind of system.
A solution could be the use of indexing, and parsing of the files and their meta-data in order to
perform fast and good search in the network.

Also, even if science article are the main goal of this tool, it could be also used in order to
deliver educational material. Manuals, exercice books (with corrections), quizzes could be also used
in a different instance of this network but would be using the same system and the same tools. This could
permits to people all over the Internet to access free and high quality (because highly reviewed)
education material.

