\section{Indexation}

In order to perform quick and accurate search, we must provide a good and efficient
way to search among a P2P system.

\subsection{Bib\TeX{} analysis and reference}

As previous work said \cite{Lawrence99digitallibraries} librairies
can completly work on a distributed network. The main information in order
to index them quickly and efficiently is the bibliography file. All article should
have a Bib\TeX like file in order to describe the content of the file without having
to parse the complete file.

BiB\TeX are already available in a lot of bibliographic system and works efficiently
in order to calculate the referencement of a precise article, and how to rank it in 
a search engine result.

Even if this system is mature, P2P systems needs to 
have efficient searching techniques. Various works already exists on this 
field (\cite{Reynolds_efficientpeer-to-peer} ,\cite{Yang02improvingsearch})
and it seems that this techniques could be implemented really easily 
into our system.

\subsection{No double utility}

Manage a large amount of articles is very complicated. The titles
are not always very precise, and some file could be actually the same.
A main interest of our system is the impossibility for a 
file to be doubled. Every file is identified by its hash and its
meta-information. Therefore there is no possibility of double. In
case of a mistake from the administrators, the distance (defined in
this article) between
an article in its slightly altered copy will be very low. This could
be a sign that something went wrong and the copy could be erased in
order to keep the system clean.
