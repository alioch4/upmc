\begin{abstract}

Reading scientific article is a daily activity for most scientists.
Even if a lot of tools are already present today, it could be very interesting
to see how P2P technologies can help in order to build a science article delivery
network completly decentralized.
This paper aims to describe what should be a P2P network of repository of scientific
content.

This paper could also be useful to people who want to design a delivering education content
system
to students all over the world.
It could also be useful for people in a company that want to use a large amount of computers
in order to avoid using a very large centralized one.

\end{abstract}
