\documentclass[twocolumns]{article}
% L'option draft est vraiment utile pour avoir un document sans 
% erreurs de OverFull UnderFull
\usepackage{multicol}
\usepackage{xltxtra}
\usepackage{wrapfig}
\usepackage[usenames]{color}
\usepackage{xunicode}
\usepackage{fontspec}
\usepackage{graphicx}
\usepackage{polyglossia}
\usepackage{setspace}
\usepackage{url}
\usepackage{appendix}
\usepackage{numprint}
\usepackage[xetex]{hyperref}

\newcommand{\BibTeX}{{\sc Bib}\TeX}


% == Metadata =============================================
\title{A Survey of Green Networking Research}
\author{Rémy Léone}
\def\keywords{upmc, green networking}
\def\subject{Fiche de lecture}

\date{\today}
\setlength{\textwidth}{39pc}

\setlength{\textheight}{54pc}

\setlength{\parindent}{1em}

\setlength{\parskip}{0pt plus 1pt}

\setlength{\oddsidemargin}{0pc}

\setlength{\marginparwidth}{0pc}

\setlength{\topmargin}{-2.5pc}

\setlength{\headsep}{20pt}

\setlength{\columnsep}{1.5pc}

\setlength\premulticols{6\baselineskip}

\hypersetup{
  bookmarks=true, % show bookmarks bar?
    pdftoolbar=true, % show Acrobat’s toolbar?
    pdfmenubar=true, % show Acrobat’s menu?
    pdffitwindow=false, % window fit to page when opened
    pdftitle={Rémy Leone - Fiche de lecture - A Survey of Green Networking Research}, % title
    pdfauthor={Rémy \textsc{Leone}}, % author
    pdfcreator={XelaTeX}, % creator of the document
    pdfproducer={Rémy \textsc{Leone}}, % producer of the document
    pdfnewwindow=false, % links in new window
    colorlinks=true, % false: boxed links; true: colored links
    linkcolor=red, % color of internal links
    citecolor=green, % color of links to bibliography
    filecolor=magenta, % color of file links
    urlcolor= blue % color of external links
}

\setdefaultlanguage{french}
\setotherlanguage{english}

\setromanfont[Mapping=tex-text]{Gentium Basic}

\begin{document}


\maketitle




%\onehalfspacing
\begin{multicols}{2}
\thispagestyle{empty}

% ==================================================================
% AVANT-PROPOS
\section{Introduction}


% ==================================================================
% CONTENU

\section*{Stratégies \& Critères}

Il existe plusieurs types de stratégies à l'oeuvre pour faire des économies
d'énergies. L'article présente 4 grands types d'approches qui peuvent permettre
de classifier les stratégies employées :

La \textbf{consolidation de ressources} est la plus simple, elle consiste à
dimmensionner en fonction des charges moyennes plutôt que pour les charges
crêtes.

La \textbf{virtualisation} permet de faire tourner plusieurs environnements
virtuels distincts sur une seule machine physique. Ainsi, il n'est plus
nécessaire d'avoir des machines différentes pour avoir des isolations logiques.
Cette technique permet d'utiliser plus efficacement une ressource physique déjà
présente.

La \textbf{connectivité sélective} permet de prévoir quelles seront les
connexions actives au sein d'un réseau en fonction du temps.  Ainsi, lorsque le
trafic sera plus faible, des liens seront temporairement endormis permettant
ainsi d'économiser de l'énergie.

Enfin l'\textbf{informatique proportionnelle} permet de concevoir une approche
dans laquelle l'augmentation de la puissance d'une machine s'obtient avec une
augmentation de l'énergie qu'elle utilise. 


\section*{Techniques}

L'article présente une méthodologie afin d'apporter des cas concrets
de systèmes à implémenter. En outre, des stratégies de mesure sont
 également introduite afin de juger de l'efficacité de ces stratégies.

%\section*{Débits adaptatifs}

L'\textbf{adaptation des débits} des liens réseau peut améliorer l'efficacité
énergétique.  Cependant, il peut occasionner des reprises difficiles et
éventuellement provoquer des phénomènes d'oscillations rendant la technique
imparfaite.  L'allocation du débit optimal sur chaque lien du réseau est un
problème très difficile à résoudre mais présente l'avantage de ne pas causer
des pertes de performances dues à des départs à froid.

%\section*{Proxy}

Les \textbf{proxys} sont des unités qui permettent de déporter le traitement
des requêtes les plus basiques afin d'éviter d'avoir à faire sortir le système
d'exploitation de son sommeil. Cette stratégie peut s'appliquer aussi bien à
l'échelle d'une machine, qu'à un réseau.

%\section*{Infrastructures à basse consommation}

La planification à l'échelle d'une \textbf{infrastructure à basse consommation}
peut apporter des résultats très significatifs.  Il existe une approche
centralisée qui a fait ses preuves dans la gestion de grille. Il existe une
approche qui se place à la bordure de la structure afin d'agréger le contenu
qui arrive dans le système. En outre, il existe également des techniques de
gestion des liens et des tables de routage afin de gérer le trafic réseau de
manière plus efficace énergétiquement.

%\section*{Logiciels \& architectures à basse consommation}

Les logiciels utilisent des bibliothèques conçues sans considérations pour les
économies d'énergie.  Ainsi, il est intéressant d'introduire des
\textbf{architectures logicielles} tel que le Pair-à-pair en intégrant des
problématiques énergétiques.  Dans cette catégorie, les progrès les plus
intéressant sont à faire dans le noyau des systèmes d'exploitation et des les
pilotes des cartes réseaux car c'est là que se trouvent les programmes les plus
executés.

%%\section*{Mesure de consommation énergétique}


La \textbf{mesure de la consommation énergétique} est essentielle afin de juger
de l'efficacité d'une stratégie. Ainsi, la mesure de la consommation
énergétique au niveau d'un noeud du réseau, d'un équipement réseau ou bien à
l'échelle d'un réseau tout entier donne des résultats pouvant évaluer les
économies énergétiques réalisées.  En outre, la combinaison de ces mesures avec
des repères pré-existants peut donner une échelle de comparaison entre les
différentes solutions possibles et leurs impacts sur les performances d'un
réseau.


% ==================================================================
% CONCLUSION
\chapter*{Conclusion générale}
\addcontentsline{toc}{chapter}{Conclusion}


Nous avons vu dans ce papier quelles étaient les techniques pour construire
des botnets resistants contre les honeypot. Le constat est assez clair, si
les administrateurs systèmes ne prennent pas le contrôle de noeud du botnet,
ils ne disposent de peu d'informations et de peu de moyens pour agir efficacement
et durablement pour la neutralisation des botnets.

Cet article sort également de sa dimension strictement technique pour toucher
un problème de fond de responsabilité judiciaire des administrateurs. Tant
que les administrateurs de parc informatique n'auront pas les mains plus libres
pour essayer de rentrer dans le botnet quitte à envoyer des flux malveillants, les
controlleurs de botnet auront toujours une longueur d'avance et les administrateurs
seront toujours condamnés à avoir une defense non offensive contre les botnet. Une
institution non gouvernementale et internationale pourrait permettre d'encadrer ces
cas houleux afin d'avoir à terme une methodologie efficace contre les botnets qui seront
toujours libres d'agir à l'encontre de tout impératif légaux.


\end{multicols}
\end{document}
