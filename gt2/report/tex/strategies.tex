\section*{Stratégies \& Critères}

Il existe plusieurs types de stratégies à l'oeuvre pour faire des économies
d'énergies. L'article présente 4 grands types d'approches qui peuvent permettre
de classifier les stratégies employées :

La \textbf{consolidation de ressources} est la plus simple, elle consiste à
dimmensionner en fonction des charges moyennes plutôt que pour les charges
crêtes.

La \textbf{virtualisation} permet de faire tourner plusieurs environnements
virtuels distincts sur une seule machine physique. Ainsi, il n'est plus
nécessaire d'avoir des machines différentes pour avoir des isolations logiques.
Cette technique permet d'utiliser plus efficacement une ressource physique déjà
présente.

La \textbf{connectivité sélective} permet de prévoir quelles seront les
connexions actives au sein d'un réseau en fonction du temps.  Ainsi, lorsque le
trafic sera plus faible, des liens seront temporairement endormis permettant
ainsi d'économiser de l'énergie.

Enfin l'\textbf{informatique proportionnelle} permet de concevoir une approche
dans laquelle l'augmentation de la puissance d'une machine s'obtient avec une
augmentation de l'énergie qu'elle utilise. 
