% Body

\begin{document}

\maketitle

\begin{frame}{Introduction}{Intérêts \& échelle}

\begin{block}{Intérêts}

\begin{itemize}
\item Économique
\item Envioronnemental
\item Recherche
\end{itemize}

\end{block}


\begin{block}{Echelle}
\begin{itemize}
\item Datacenter (Consommation et refroidissement de taille industriel)
\item Embarqué (L'énergie est un critère de pérénitté du système)
\item Informatique individuelle
\end{itemize}
\end{block}

\end{frame}

\begin{frame}{Point de vue \& Objectifs}{}

\begin{block}{Définition du ``green computing''}
\begin{itemize}
\item Point de vue de l'ingénieur
\item Point de vue du régulateur (Gouvernements)
\item Point de vue économique
\item Point de vue environnemental
\end{itemize}
\end{block}


\begin{block}{Application dans les réseaux \& Objectif}
\begin{itemize}
\item Etude des réseaux filaires
\item Hub \& Switch représentaient (80 \% de la consommation énergétique de l'internet en 2002)
\item Combiner qualité du réseau et économies d'énergies.
\end{itemize}
\end{block}

\end{frame}

\begin{frame}{Stratégies pour les économies d'énergies}{}

\begin{block}{Consolidation de ressources}
Prévoir les ressources nécessaires
\end{block}
\begin{block}{Virtualisation}
Plusieurs machines sur une seule
\end{block}
\begin{block}{Connectivité selective}
Prévoir les connexions nécessaires
\end{block}
\begin{block}{Proportionnal computing}
Consommer en fonction des ressources mobilisées
\end{block}
\end{frame}

\begin{frame}{Critères}
\begin{block}{Critères d'évaluation d'une stratégie}
\begin{itemize}
\item Echelles de temps (Mise à jour de la politique)
\item Portée des actions (Local ou à l'échelle d'un réseau)
\item Couches OSI (A quel(s) niveau(x) le système se met en place)
\item Entrées (Plan d'action ou Réaction à l'environnement)
\item Approches (Analyse de trafic, Modèles, Simulation, Hardware, Software, \ldots)
\end{itemize}
\end{block}

\end{frame}

\begin{frame}{Débit adaptatif}{\textit{Adaptative Link Rate}}

\begin{block}{Mode sommeil}
\begin{itemize}
\item Facile à mettre en oeuvre et mesure facile
\item Temps entre chaque paquets
\item Reprise difficile (Possibilité d'oscillations)
\end{itemize}
\end{block}


\begin{block}{Débit adaptatif}
\begin{itemize}
\item Trouver l'optimal est NP-difficile
\item Gain d'énergie intéressant (5 \% 10Mb/s vers 1 Gb/s)
\item Plus robuste en cas de crêtes (Pas de départ à froid)
\end{itemize}
\end{block}
\end{frame}



\begin{frame}{Proxy}
\begin{block}{A l'échelle locale}
\begin{itemize}
\item Répondre aux requêtes basiques (90 \% du temps efficace)
\item Connectivité sélective
\item Ne peut pas gérer les requêtes complexes
\end{itemize}
\end{block}

\begin{block}{A l'échelle d'un réseau}
\begin{itemize}
\item Mis aux portes d'un réseau
\item Requêtes un peu plus complexes que les proxy à échelle locale
\item Evalué dans un contexte pair à pair avec de bons résultats
\end{itemize}
\end{block}
\end{frame}

\begin{frame}{Infrastructure à basse consommation}{\textit{Energy-aware infrastructure}}
\begin{block}{Approches générales}
\begin{itemize}
\item Plannification centralisée de l'utilisation des ressources (Grid)
\item Agrégation à la bordure (Utile dans les réseaux optiques)
\end{itemize}
\end{block}

\begin{block}{Approche dans les routeurs}
\begin{itemize}
\item Contrainte de qualité de service à maintenir
\item Attention au re-calcul des chemins et aux synchronisations (OSPF \& IBGP)
\end{itemize}
\end{block}

\end{frame}

\begin{frame}{Logiciels et architecture à basse consommation}{\textit{Energy-aware software and applications}}

\begin{block}{Espace utilisateur}
\begin{itemize}
\item Applications écologiques (Pair à pair, Modèles de programmation)
\end{itemize}
\end{block}

\begin{block}{Espace noyau}
\begin{itemize}
\item Protocoles de communications à améliorer
\item Interface réseaux à améliorer
\end{itemize}
\end{block}

\begin{block}{Architecture inspirée du sans-fil}
\begin{itemize}
\item Compromis entre portée et équipements en fonctionnements
\item Lien entre puissance electrique et débit
\item Penser distribué
\end{itemize}
\end{block}


\end{frame}


\begin{frame}{Mesure de consommation énergétique}
\begin{itemize}
\item Nécessaire pour juger de l'efficacité d'une stratégie.
\end{itemize}

\begin{block}{Mesures en différents points}
\begin{itemize}
\item Noeud d'un réseau
\item Équipement de réseau (Switch, routeur, \ldots)
\item Mesure d'un réseau tout entier
\end{itemize}
\end{block}

\begin{block}{Benchmark}
\begin{itemize}
\item Applications
\item Noeuds
\item Réseaux
\end{itemize}
\end{block}
\end{frame}



\begin{frame}{Conclusion}
\begin{block}{Catégories}
\begin{itemize}
\item Adaptative link rate
\item Interface proxying
\item Energy aware infrastructure
\item Energy aware applications
\end{itemize}
\end{block}

\begin{block}{Champs de recherche}
\begin{itemize}
\item La mesure énérgétique, la modélisation
\item La cohabitation de la qualité de service et des contraites énergétiques
\end{itemize}
\end{block}
\end{frame}

\end{document}
